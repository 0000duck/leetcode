\chapter{细节实现题}
这类题目不考特定的算法,纯粹考察写代码的熟练度。


\section{Two Sum} %%%%%%%%%%%%%%%%%%%%%%%%%%%%%%
\label{sec:wordladder}


\subsubsection{描述}
Given an array of integers, find two numbers such that they add up to a specific target number.

The function twoSum should return indices of the two numbers such that they add up to the target, where index1 must be less than index2. Please note that your returned answers (both index1 and index2) are not zero-based.

You may assume that each input would have exactly one solution.

Input:  \code{numbers=\{2, 7, 11, 15\}, target=9}

Output: \code{index1=1, index2=2}


\subsubsection{分析}
方法1:暴力,复杂度$O(n^2)$,会超时

方法2:hash。用一个哈希表,存储每个数对应的下标,复杂度$O(n)$


\subsubsection{代码}
\begin{Code}
//LeetCode, Two Sum
// 方法1:暴力,O(n^2)
// 方法2:hash。用一个哈希表,存储每个数对应的下标,复杂度O(n),代码如下,
class Solution {
public:
    vector<int> twoSum(vector<int> &numbers, int target) {
        unordered_map<int, int> mapping;
        vector<int> result;
        for (int i = 0; i < numbers.size(); i++) {
            mapping[numbers[i]] = i;
        }
        for (int i = 0; i < numbers.size(); i++) {
            const int gap = target - numbers[i];
            if (mapping.find(gap) != mapping.end()) {
                result.push_back(i + 1);
                result.push_back(mapping[gap] + 1);
                break;
            }
        }
        return result;
    }
};
\end{Code}


\subsubsection{相关题目}

\begindot
\item 无
\myenddot


\section{Insert Interval} %%%%%%%%%%%%%%%%%%%%%%%%%%%%%%
\label{sec:insert-interval}


\subsubsection{描述}
Given a set of non-overlapping intervals, insert a new interval into the intervals (merge if necessary).

You may assume that the intervals were initially sorted according to their start times.

Example 1:
Given intervals \code{[1,3],[6,9]}, insert and merge \code{[2,5]} in as \code{[1,5],[6,9]}.

Example 2:
Given \code{[1,2],[3,5],[6,7],[8,10],[12,16]}, insert and merge \code{[4,9]} in as \code{[1,2],[3,10],[12,16]}.

This is because the new interval \code{[4,9]} overlaps with \code{[3,5],[6,7],[8,10]}.


\subsubsection{分析}
无


\subsubsection{代码}
\begin{Code}
struct Interval {
    int start;
    int end;
    Interval() : start(0), end(0) { }
    Interval(int s, int e) : start(s), end(e) { }
};
 
//LeetCode, Insert Interval
class Solution {
public:
    vector<Interval> insert(vector<Interval> &intervals, Interval newInterval) {
        vector<Interval>::iterator it = intervals.begin();
        while (it != intervals.end()) {
            if (newInterval.end < it->start) {
                intervals.insert(it, newInterval);
                return intervals;
            } else if (newInterval.start > it->end) {
                it++;
                continue;
            } else {
                newInterval.start = min(newInterval.start, it->start);
                newInterval.end = max(newInterval.end, it->end);
                it = intervals.erase(it);
            }
        }
        intervals.insert(intervals.end(), newInterval);
        return intervals;
    }
};
\end{Code}


\subsubsection{相关题目}

\begindot
\item Merge Intervals,见 \S \ref{sec:merge-intervals}
\myenddot


\section{Merge Intervals} %%%%%%%%%%%%%%%%%%%%%%%%%%%%%%
\label{sec:merge-intervals}


\subsubsection{描述}
Given a collection of intervals, merge all overlapping intervals.

For example,
Given \code{[1,3],[2,6],[8,10],[15,18]},
return \code{[1,6],[8,10],[15,18]}


\subsubsection{分析}
复用一下Insert Intervals的解法即可,创建一个新的interval集合,然后每次从旧的里面取一个interval出来,然后插入到新的集合中。


\subsubsection{代码}
\begin{Code}
struct Interval {
    int start;
    int end;
    Interval() : start(0), end(0) { }
    Interval(int s, int e) : start(s), end(e) { }
};
 
//LeetCode, Merge Interval
//复用一下Insert Intervals的解法即可
class Solution {
public:
    vector<Interval> merge(vector<Interval> &intervals) {
        vector<Interval> result;
        for (int i = 0; i < intervals.size(); i++) {
            insert(result, intervals[i]);
        }
        return result;
    }
private:
    vector<Interval> insert(vector<Interval> &intervals, Interval newInterval) {
        vector<Interval>::iterator it = intervals.begin();
        while (it != intervals.end()) {
            if (newInterval.end < it->start) {
                intervals.insert(it, newInterval);
                return intervals;
            } else if (newInterval.start > it->end) {
                it++;
                continue;
            } else {
                newInterval.start = min(newInterval.start, it->start);
                newInterval.end = max(newInterval.end, it->end);
                it = intervals.erase(it);
            }
        }
        intervals.insert(intervals.end(), newInterval);
        return intervals;
    }
};
\end{Code}


\subsubsection{相关题目}

\begindot
\item Insert Interval,见 \S \ref{sec:insert-interval}
\myenddot


\section{Minimum Window Substring} %%%%%%%%%%%%%%%%%%%%%%%%%%%%%%
\label{sec:minimum-window-substring}


\subsubsection{描述}
Given a string $S$ and a string $T$, find the minimum window in $S$ which will contain all the characters in $T$ in complexity $O(n)$.

For example, \code{S = "ADOBECODEBANC", T = "ABC"}

Minimum window is \code{"BANC"}.

Note:
\begindot
\item If there is no such window in $S$ that covers all characters in $T$, return the emtpy string \code{""}.
\item If there are multiple such windows, you are guaranteed that there will always be only one unique minimum window in $S$.
\myenddot


\subsubsection{分析}
双指针,动态维护一个区间。尾指针不断往后扫,当扫到有一个窗口包含了所有$T$的字符后,然后再收缩头指针,直到不能再收缩为止。最后记录所有可能的情况中窗口最小的


\subsubsection{代码}
\begin{Code}
// LeetCode, Minimum Window Substring
// 双指针,动态维护一个区间。尾指针不断往后扫,当扫到有一个窗口包含了所有T的字符
// 后,然后再收缩头指针,直到不能再收缩为止。最后记录所有可能的情况中窗口最小的
class Solution {
public:
    string minWindow(string S, string T) {
        if (S.empty()) return "";
        if (S.size() < T.size()) return "";

        const int ASCII_MAX = 256;
        int appeared_count[ASCII_MAX];
        int expected_count[ASCII_MAX];
        std::fill(appeared_count, appeared_count + ASCII_MAX, 0);
        std::fill(expected_count, expected_count + ASCII_MAX, 0);

        for (size_t i = 0; i < T.size(); i++) expected_count[T[i]]++;

        int minWidth = INT_MAX, min_start = 0;  // 窗口大小,起点
        int wnd_start = 0;
        int appeared = 0;  // 完整包含了一个T
        //尾指针不断往后扫
        for (size_t wnd_end = 0; wnd_end < S.size(); wnd_end++) {
            if (expected_count[S[wnd_end]] > 0)  {  // this char is a part of T
                appeared_count[S[wnd_end]]++;
                if (appeared_count[S[wnd_end]] <= expected_count[S[wnd_end]])
                    appeared++;
            }
            if (appeared == T.size()) {  // 完整包含了一个T
                // 收缩头指针
                while (appeared_count[S[wnd_start]] > expected_count[S[wnd_start]]
                        || expected_count[S[wnd_start]] == 0) {
                    appeared_count[S[wnd_start]]--;
                    wnd_start++;
                }
                if (minWidth > (wnd_end - wnd_start + 1)) {
                    minWidth = wnd_end - wnd_start + 1;
                    min_start = wnd_start;
                }
            }
        }

        if (minWidth == INT_MAX) return "";
        else return S.substr(min_start, minWidth);
    }
};
\end{Code}


\subsubsection{相关题目}

\begindot
\item 无
\myenddot


\section{Multiply Strings} %%%%%%%%%%%%%%%%%%%%%%%%%%%%%%
\label{sec:multiply-strings}


\subsubsection{描述}
Given two numbers represented as strings, return multiplication of the numbers as a string.

Note: The numbers can be arbitrarily large and are non-negative.


\subsubsection{分析}
高精度乘法。

常见的做法是将字符转化为一个int,一一对应,形成一个int数组。但是这样很浪费空间,一个int32的最大值是$2^{31}-1=2147483647$,可以与9个字符对应,由于有乘法,减半,则至少可以与4个字符一一对应。一个int64可以与9个字符对应。


\subsubsection{代码}

\begin{Code}
// LeetCode, Multiply Strings
// 一个字符对应一个int
typedef vector<char> bigint;

bigint make_bigint(string const& repr) {
    bigint n;
    transform(repr.rbegin(), repr.rend(), back_inserter(n), [](char c) { return c - '0'; });
    return move(n);
}

string to_string(bigint const& n) {
    string str;
    transform(find_if(n.rbegin(), prev(n.rend()), [](char c) { return c > '\0'; }),
              n.rend(), back_inserter(str), [](char c) { return c + '0'; });
    return move(str);
}

bigint operator*(bigint const& x, bigint const& y) {
    bigint z(x.size() + y.size());

    for (size_t i = 0; i < x.size(); ++i)
        for (size_t j = 0; j < y.size(); ++j) {
            z[i + j] += x[i] * y[j];
            z[i + j + 1] += z[i + j] / 10;
            z[i + j] %= 10;
        }

    return move(z);
}

class Solution {
public:
    string multiply(string num1, string num2) {
        return to_string(make_bigint(num1) * make_bigint(num2));
    }
};
\end{Code}

\begin{Code}
// LeetCode, Multiply Strings
// 9个字符对应一个int64_t
/** 大整数类. */
class BigInt {
public:
    /**
     * @brief 构造函数,将字符串转化为大整数.
     * @param[in] s 输入的字符串
     * @return 无
     */
    BigInt(string s) {
        vector<int64_t> result;
        result.reserve(s.size() / RADIX_LEN + 1);

        for (int i = s.size(); i > 0; i -= RADIX_LEN) {  // [i-RADIX_LEN, i)
            int temp = 0;
            const int low = max(i - RADIX_LEN, 0);
            for (int j = low; j < i; j++) {
                temp = temp * 10 + s[j] - '0';
            }
            result.push_back(temp);
        }
        elems = result;
    }
    /**
     * @brief 将整数转化为字符串.
     * @return 字符串
     */
    string toString() {
        stringstream result;
        bool started = false; // 用于跳过前导0
        for (auto i = elems.rbegin(); i != elems.rend(); i++) {
            if (started) { // 如果多余的0已经都跳过,则输出
                result << std::setw(RADIX_LEN) << std::setfill('0') << *i;
            } else {
                result << *i;
                started = true; // 碰到第一个非0的值,就说明多余的0已经都跳过
            }
        }

        if (!started) return "0"; // 当x全为0时
        else return result.str();
    }

    /**
     * @brief 大整数乘法.
     * @param[in] x x
     * @param[in] y y
     * @return 大整数
     */
    static BigInt multiply(const BigInt &x, const BigInt &y) {
        vector<int64_t> z(x.elems.size() + y.elems.size(), 0);

        for (size_t i = 0; i < y.elems.size(); i++) {
            for (size_t j = 0; j < x.elems.size(); j++) { // 用y[i]去乘以x的各位
                //  两数第i, j位相乘,累加到结果的第i+j位
                z[i + j] += y.elems[i] * x.elems[j];

                if (z[i + j] >= BIGINT_RADIX) { //  看是否要进位
                    z[i + j + 1] += z[i + j] / BIGINT_RADIX; //  进位
                    z[i + j] %= BIGINT_RADIX;
                }
            }
        }
        while (z.back() == 0) z.pop_back();  // 没有进位,去掉最高位的0
        return BigInt(z);
    }

private:
    typedef long long int64_t;
    /** 一个数组元素对应9个十进制位,即数组是亿进制的
     * 因为 1000000000 * 1000000000 没有超过 2^63-1
     */
    const static int BIGINT_RADIX = 1000000000;
    const static int RADIX_LEN = 9;
    /** 万进制整数. */
    vector<int64_t> elems;
    BigInt(const vector<int64_t> num) : elems(num) {}
};


class Solution {
public:
    string multiply(string num1, string num2) {
        BigInt x(num1);
        BigInt y(num2);
        return BigInt::multiply(x, y).toString();
    }
};
\end{Code}


\subsubsection{相关题目}

\begindot
\item 无
\myenddot


\section{Substring with Concatenation of All Words} %%%%%%%%%%%%%%%%%%%%%%%%%%%%%%
\label{sec:substring-with-concatenation-of-all-words}


\subsubsection{描述}
You are given a string, $S$, and a list of words, $L$, that are all of the same length. Find all starting indices of substring(s) in $S$ that is a concatenation of each word in $L$ exactly once and without any intervening characters.

For example, given: 
\begin{Code}
S: "barfoothefoobarman"
L: ["foo", "bar"]
\end{Code}

You should return the indices: \code{[0,9]}.(order does not matter).


\subsubsection{分析}
无


\subsubsection{代码}
\begin{Code}
// LeetCode, Substring with Concatenation of All Words
class Solution {
public:
    vector<int> findSubstring(string s, vector<string>& dict)
    {
        size_t wordLength = dict.front().length();
        size_t catLength = wordLength * dict.size();
        vector<int> result;

        if (s.length() < catLength) return result;

        unordered_map<string, int> wordCount;

        for (auto const& word : dict) ++wordCount[word];

        for (auto i = begin(s); i <= prev(end(s), catLength); ++i) {
            unordered_map<string, int> unused(wordCount);

            for (auto j = i; j != next(i, catLength); j += wordLength) {
                auto pos = unused.find(string(j, next(j, wordLength)));

                if (pos == unused.end() or pos->second == 0) break;

                if (--pos->second == 0) unused.erase(pos);
            }

            if (unused.size() == 0) result.push_back(distance(begin(s), i));
        }

        return result;
    }
};
\end{Code}


\subsubsection{相关题目}

\begindot
\item 无
\myenddot


\section{Pascal's Triangle} %%%%%%%%%%%%%%%%%%%%%%%%%%%%%%
\label{sec:pascals-triangle}


\subsubsection{描述}
Given $numRows$, generate the first $numRows$ of Pascal's triangle.

For example, given $numRows = 5$,

Return
\begin{Code}
[
     [1],
    [1,1],
   [1,2,1],
  [1,3,3,1],
 [1,4,6,4,1]
]
\end{Code}


\subsubsection{分析}
本题可以用队列,计算下一行时,给上一行左右各加一个0,然后下一行的每个元素,就等于左上角和右上角之和。

另一种思路,下一行第一个元素和最后一个元素赋值为1,中间的每个元素,等于上一行的左上角和右上角元素之和。


\subsubsection{代码}

\begin{Code}
// LeetCode, Pascal's Triangle
class Solution {
public:
    vector<vector<int> > generate(int numRows) {
        vector<vector<int> > result;
        if(numRows == 0) return result;

        result.push_back(vector<int>(1,1)); //first row

        for(int i = 2; i <= numRows; ++i) {
            vector<int> thisrow(i,1);  // 本行
            vector<int> &lastrow = result[i-2];  // 上一行

            for(int j = 1; j < i - 1; ++j) {
                thisrow[j] = lastrow[j-1] + lastrow[j]; // 左上角和右上角之和
            }
            result.push_back(thisrow);
        }
        return result;
    }
};
\end{Code}


\subsubsection{相关题目}
\begindot
\item Pascal's Triangle II,见 \S \ref{sec:pascals-triangle-ii}
\myenddot


\section{Pascal's Triangle II} %%%%%%%%%%%%%%%%%%%%%%%%%%%%%%
\label{sec:pascals-triangle-ii}


\subsubsection{描述}
Given an index $k$, return the $k^{th}$ row of the Pascal's triangle.

For example, given $k = 3$,

Return \code{[1,3,3,1]}.

Note: Could you optimize your algorithm to use only $O(k)$ extra space?


\subsubsection{分析}
滚动数组。


\subsubsection{代码}

\begin{Code}
// LeetCode, Pascal's Triangle
// 滚动数组
class Solution {
public:
    vector<int> getRow(int rowIndex) {
        vector<int> result;
        result.resize(rowIndex + 2);
        result.assign(result.size(), 0);

        result[1] = 1;
        for (int i = 0; i < rowIndex; i++) {
            for (int j = rowIndex + 1; j > 0; j--) {
                result[j] = result[j - 1] + result[j];
            }
        }
        result.erase(result.begin());
        return result;
    }
};
\end{Code}


\subsubsection{相关题目}
\begindot
\item Pascal's Triangle,见 \S \ref{sec:pascals-triangle}
\myenddot


\section{Spiral Matrix} %%%%%%%%%%%%%%%%%%%%%%%%%%%%%%
\label{sec:spiral-matrix}


\subsubsection{描述}
Given a matrix of $m \times n$ elements ($m$ rows, $n$ columns), return all elements of the matrix in spiral order.

For example,
Given the following matrix:
\begin{Code}
[
 [ 1, 2, 3 ],
 [ 4, 5, 6 ],
 [ 7, 8, 9 ]
]
\end{Code}
You should return \fn{[1,2,3,6,9,8,7,4,5]}.


\subsubsection{分析}
无

\subsubsection{代码}
\begin{Code}
// LeetCode, Spiral Matrix
class Solution {
public:
    vector<int> spiralOrder(vector<vector<int> > &matrix) {
        const int m = matrix.size();
        if (m == 0) return vector<int>();
        const int n = matrix[0].size();
        if (n == 0) return vector<int>();

        vector<int> result(m * n, 0);

        int layer, index;
        for (layer = min(m, n), index = 0; layer > 1; layer -= 2) {
            const int offset = (min(m, n) - layer) / 2;
            // left to right
            for (int i = offset; i < n - offset - 1; i++)
                result[index++] = matrix[offset][i];
            // top to bottom
            for (int i = offset; i < m - offset - 1; i++)
                result[index++] = matrix[i][n - offset - 1];
            // right to left
            for (int i = n - offset - 1; i > offset; i--)
                result[index++] = matrix[m - offset - 1][i];
            // bottom to top
            for (int i = m - offset - 1; i > offset; i--)
                result[index++] = matrix[i][offset];
        }

        if (layer == 1) {  // 最后一行
            if (m < n)
                for (int i = m / 2; i < n - m / 2; i++)
                    result[index++] = matrix[m / 2][i];
            else
                for (int i = n / 2; i < m - n / 2; i++)
                       result[index++] = matrix[i][n / 2];
        }
        return result;
    }
};
\end{Code}


\subsubsection{相关题目}
\begindot
\item Spiral Matrix II ,见 \S \ref{sec:spiral-matrix-ii}
\myenddot


\section{Spiral Matrix II} %%%%%%%%%%%%%%%%%%%%%%%%%%%%%%
\label{sec:spiral-matrix-ii}


\subsubsection{描述}
Given an integer $n$, generate a square matrix filled with elements from 1 to $n^2$ in spiral order.

For example,
Given $n = 3$,

You should return the following matrix:
\begin{Code}
[
 [ 1, 2, 3 ],
 [ 8, 9, 4 ],
 [ 7, 6, 5 ]
]
\end{Code}


\subsubsection{分析}
这题比上一题要简单。


\subsubsection{代码}
\begin{Code}
// LeetCode, Spiral Matrix II
class Solution {
public:
    vector<vector<int> > generateMatrix(int n) {
        if (n == 0) return vector<vector<int> >();

        vector<vector<int> > matrix(n, vector<int>(n, 0));

        int num = 1, layer;
        for (layer = n; layer > 1; layer -= 2) {
            const int offset = (n - layer) / 2;
            // left to right
            for (int i = offset; i < n - offset - 1; i++)
                matrix[offset][i] = num++;
            // top to bottom
            for (int i = offset; i < n - offset - 1; i++)
                matrix[i][n - offset - 1] = num++;
            // right to left
            for (int i = n - offset - 1; i > offset; i--)
                matrix[n - offset - 1][i] = num++;
            // bottom to top
            for (int i = n - offset - 1; i > offset; i--)
                matrix[i][offset] = num++;
        }
        if (layer == 1) matrix[n / 2][n / 2] = num;

        return matrix;
    }
};
\end{Code}


\subsubsection{相关题目}
\begindot
\item Spiral Matrix ,见 \S \ref{sec:spiral-matrix}
\myenddot
