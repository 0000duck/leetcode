\chapter{广度优先搜索}
当题目看不出任何规律,既不能用分治,贪心,也不能用动规时,这时候万能方法——搜索,
就派上用场了。搜索分为广搜和深搜,广搜里面又有普通广搜,双向广搜,A*搜索等。
深搜里面又有普通深搜,回溯法等。

广搜和深搜非常类似(除了在扩展节点这部分不一样),二者有相同的框架,如何表示状态?
如何扩展状态?如何判重?尤其是判重,解决了这个问题,基本上整个问题就解决了。


\section{Word Ladder} %%%%%%%%%%%%%%%%%%%%%%%%%%%%%%
\label{sec:wordladder}


\subsubsection{描述}
Given two words (start and end), and a dictionary, find the length of shortest transformation sequence from start to end, such that:
\begindot
\item Only one letter can be changed at a time
\item Each intermediate word must exist in the dictionary
\myenddot

For example, Given:

\begin{Code}
start = "hit"
end = "cog"
dict = ["hot","dot","dog","lot","log"]
\end{Code}
As one shortest transformation is \code{"hit" -> "hot" -> "dot" -> "dog" -> "cog"}, return its length $5$.

Note:
\begindot
\item Return 0 if there is no such transformation sequence.
\item All words have the same length.
\item All words contain only lowercase alphabetic characters.
\myenddot


\subsubsection{分析}


\subsubsection{代码}
\begin{Code}
//LeetCode, Word Ladder
class Solution {
public:
    int ladderLength(string start, string end, unordered_set<string> &dict) {
        if (start.size() != end.size()) return 0;
        if (start.empty() || end.empty()) return 1; return 0;
        int level = 1;  // 层次
        queue<string> queToPush, queToPop;

        queToPop.push(start);
        while (dict.size() > 0 && !queToPop.empty()) {
            while (!queToPop.empty()) {
                string str(queToPop.front());
                queToPop.pop();
                for (int i = 0; i < str.size(); i++) {
                    for (char j = 'a'; j <= 'z'; j++) {
                        if (j == str[i]) continue;

                        const char temp = str[i];
                        str[i] = j;
                        if (str == end) return level + 1; //找到了

                        if (dict.count(str) > 0) {
                            queToPush.push(str);
                            dict.erase(str); // 删除该单词,防止死循环
                        }
                        str[i] = temp; // 恢复该单词
                    }
                }
            }
            swap(queToPush, queToPop); //!!! 交换两个队列
            level++;
        }
        return 0; // 所有单词已经用光,还是找不到通向目标单词的路径
    }
};
\end{Code}


\subsubsection{相关题目}

\begindot
\item Word Ladder II,见 \S \ref{sec:wordladderii}
\myenddot


\section{Word Ladder II} %%%%%%%%%%%%%%%%%%%%%%%%%%%%%%
\label{sec:wordladderii}


\subsubsection{描述}
Given two words (start and end), and a dictionary, find all shortest transformation sequence(s) from start to end, such that:
\begindot
\item Only one letter can be changed at a time
\item Each intermediate word must exist in the dictionary
\myenddot

For example, Given:
\begin{Code}
start = "hit"
end = "cog"
dict = ["hot","dot","dog","lot","log"]
\end{Code}
Return
\begin{Code}
[
    ["hit","hot","dot","dog","cog"],
    ["hit","hot","lot","log","cog"]
]
\end{Code}

Note:
\begindot
\item All words have the same length.
\item All words contain only lowercase alphabetic characters.
\myenddot


\subsubsection{分析}
跟 Word Ladder比,这题是求路径本身,不是路径长度,也是BFS,略微麻烦点


\subsubsection{代码}

\begin{Code}
//LeetCode, Word Ladder II
//跟 Word Ladder比,这题是求路径本身,不是路径长度,也是BFS,略微麻烦点
class Solution {
public:
    std::vector<std::vector<std::string> > findLadders(std::string start,
            std::string end, std::unordered_set<std::string> &dict) {
        result_.clear();
        std::unordered_map<std::string, std::vector<std::string>> prevMap;

        for (auto iter = dict.begin(); iter != dict.end(); ++iter) {
            prevMap[*iter] = std::vector<std::string>();
        }

        std::vector<std::unordered_set<std::string>> candidates(2);

        int current = 0;
        int previous = 1;

        candidates[current].insert(start);

        while (true) {
            current = !current;
            previous = !previous;

            // 从dict中删除previous中的单词,避免自己指向自己
            for (auto iter = candidates[previous].begin();
                    iter != candidates[previous].end(); ++iter) {
                dict.erase(*iter);
            }

            candidates[current].clear();

            for (auto iter = candidates[previous].begin();
                    iter != candidates[previous].end(); ++iter) {
                for (size_t pos = 0; pos < iter->size(); ++pos) {
                    std::string word = *iter;
                    for (int i = 'a'; i <= 'z'; ++i) {
                        if (word[pos] == i) continue;

                        word[pos] = i;

                        if (dict.count(word) > 0) {
                            prevMap[word].push_back(*iter);
                            candidates[current].insert(word);
                        }
                    }
                }
            }

            if (candidates[current].size() == 0) return result_;  // 此题无解
            if (candidates[current].count(end)) break; // 没看懂
        }

        std::vector<std::string> path;
        GeneratePath(prevMap, path, end);

        return result_;
    }

private:
    std::vector<std::vector<std::string>> result_;

    void GeneratePath(
            std::unordered_map<std::string, std::vector<std::string>> &prevMap,
            std::vector<std::string>& path, const std::string& word) {
        if (prevMap[word].size() == 0) {
            path.push_back(word);
            std::vector<std::string> curPath = path;
            reverse(curPath.begin(), curPath.end());
            result_.push_back(curPath);
            path.pop_back();
            return;
        }

        path.push_back(word);
        for (auto iter = prevMap[word].begin(); iter != prevMap[word].end();
                ++iter) {
            GeneratePath(prevMap, path, *iter);
        }
        path.pop_back();
    }
};
\end{Code}


\subsubsection{相关题目}

\begindot
\item Word Ladder,见 \S \ref{sec:wordladder}
\myenddot
