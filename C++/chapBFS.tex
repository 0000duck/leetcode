\chapter{广度优先搜索}
当题目看不出任何规律,既不能用分治,贪心,也不能用动规时,这时候万能方法——搜索,
就派上用场了。搜索分为广搜和深搜,广搜里面又有普通广搜,双向广搜,A*搜索等。
深搜里面又有普通深搜,回溯法等。

广搜和深搜非常类似(除了在扩展节点这部分不一样),二者有相同的框架,如何表示状态?
如何扩展状态?如何判重?尤其是判重,解决了这个问题,基本上整个问题就解决了。


\section{Word Ladder} %%%%%%%%%%%%%%%%%%%%%%%%%%%%%%
\label{sec:word-ladder}


\subsubsection{描述}
Given two words (start and end), and a dictionary, find the length of shortest transformation sequence from start to end, such that:
\begindot
\item Only one letter can be changed at a time
\item Each intermediate word must exist in the dictionary
\myenddot

For example, Given:

\begin{Code}
start = "hit"
end = "cog"
dict = ["hot","dot","dog","lot","log"]
\end{Code}
As one shortest transformation is \code{"hit" -> "hot" -> "dot" -> "dog" -> "cog"}, return its length $5$.

Note:
\begindot
\item Return 0 if there is no such transformation sequence.
\item All words have the same length.
\item All words contain only lowercase alphabetic characters.
\myenddot


\subsubsection{分析}
求最短路径,用广搜。


\subsubsection{单队列}
\begin{Code}
//LeetCode, Word Ladder
// 时间复杂度O(n),空间复杂度O(n)
struct state_t {
    string word;
    int level;

    state_t() { word = ""; level = 0; }
    state_t(const string& word, int level) {
        this->word = word;
        this->level = level;
    }

    bool operator==(const state_t &other) const {
        return this->word == other.word;
    }
};

namespace std {
    template<> struct hash<state_t> {
    public:
        size_t operator()(const state_t& s) const {
            return str_hash(s.word);
        }
    private:
        std::hash<std::string> str_hash;
    };
}


class Solution {
public:
    int ladderLength(const string& start, const string &end,
            const unordered_set<string> &dict) {
        queue<state_t> q;
        unordered_set<state_t> visited;  // 判重

        auto state_is_valid = [&](const state_t& s) {
            return dict.find(s.word) != dict.end() || s.word == end;
        };
        auto state_is_target = [&](const state_t &s) {return s.word == end; };
        auto state_extend = [&](const state_t &s) {
            unordered_set<state_t> result;

            for (size_t i = 0; i < s.word.size(); ++i) {
                state_t new_state(s.word, s.level + 1);
                for (char c = 'a'; c <= 'z'; c++) {
                    // 防止同字母替换
                    if (c == new_state.word[i]) continue;

                    swap(c, new_state.word[i]);

                    if (state_is_valid(new_state) &&
                        visited.find(new_state) == visited.end()) {
                        result.insert(new_state);
                    }
                    swap(c, new_state.word[i]); // 恢复该单词
                }
            }

            return result;
        };

        state_t start_state(start, 0);
        q.push(start_state);
        visited.insert(start_state);
        while (!q.empty()) {
            // 千万不能用 const auto&,pop() 会删除元素,
            // 引用就变成了悬空引用
            const auto state = q.front();
            q.pop();

            if (state_is_target(state)) {
                return state.level + 1;
            }

            const auto& new_states = state_extend(state);
            for (const auto& new_state : new_states) {
                q.push(new_state);
                visited.insert(new_state);
            }
        }
        return 0;
    }
};
\end{Code}


\subsubsection{双队列}
\begin{Code}
//LeetCode, Word Ladder
// 时间复杂度O(n),空间复杂度O(n)
class Solution {
public:
    int ladderLength(const string& start, const string &end,
            const unordered_set<string> &dict) {
        queue<string> current, next;    // 当前层,下一层
        unordered_set<string> visited;  // 判重

        int level = -1;  // 层次

        auto state_is_valid = [&](const string& s) {
            return dict.find(s) != dict.end() || s == end;
        };
        auto state_is_target = [&](const string &s) {return s == end;};
        auto state_extend = [&](const string &s) {
            unordered_set<string> result;

            for (size_t i = 0; i < s.size(); ++i) {
                string new_word(s);
                for (char c = 'a'; c <= 'z'; c++) {
                    // 防止同字母替换
                    if (c == new_word[i]) continue;

                    swap(c, new_word[i]);

                    if (state_is_valid(new_word) &&
                        visited.find(new_word) == visited.end()) {
                        result.insert(new_word);
                    }
                    swap(c, new_word[i]); // 恢复该单词
                }
            }

            return result;
        };

        current.push(start);
        visited.insert(start);
        while (!current.empty()) {
            ++level;
            while (!current.empty()) {
                // 千万不能用 const auto&,pop() 会删除元素,
                // 引用就变成了悬空引用
                const auto state = current.front();
                current.pop();

                if (state_is_target(state)) {
                    return level + 1;
                }

                const auto& new_states = state_extend(state);
                for (const auto& new_state : new_states) {
                    next.push(new_state);
                    visited.insert(new_state);
                }
            }
            swap(next, current);
        }
        return 0;
    }
};
\end{Code}


\subsubsection{相关题目}

\begindot
\item Word Ladder II,见 \S \ref{sec:word-ladder-ii}
\myenddot


\section{Word Ladder II} %%%%%%%%%%%%%%%%%%%%%%%%%%%%%%
\label{sec:word-ladder-ii}


\subsubsection{描述}
Given two words (start and end), and a dictionary, find all shortest transformation sequence(s) from start to end, such that:
\begindot
\item Only one letter can be changed at a time
\item Each intermediate word must exist in the dictionary
\myenddot

For example, Given:
\begin{Code}
start = "hit"
end = "cog"
dict = ["hot","dot","dog","lot","log"]
\end{Code}
Return
\begin{Code}
[
    ["hit","hot","dot","dog","cog"],
    ["hit","hot","lot","log","cog"]
]
\end{Code}

Note:
\begindot
\item All words have the same length.
\item All words contain only lowercase alphabetic characters.
\myenddot


\subsubsection{分析}
跟 Word Ladder比,这题是求路径本身,不是路径长度,也是BFS,略微麻烦点。

求一条路径和求所有路径有很大的不同,求一条路径,每个状态节点只需要记录一个前驱即可;求所有路径时,有的状态节点可能有多个父节点,即要记录多个前驱。

如果当前路径长度已经超过当前最短路径长度,可以中止对该路径的处理,因为我们要找的是最短路径。


\subsubsection{单队列}

\begin{Code}
//LeetCode, Word Ladder II
// 时间复杂度O(n),空间复杂度O(n)
struct state_t {
    string word;
    int level;

    state_t() { word = ""; level = 0; }
    state_t(const string& word, int level) {
        this->word = word;
        this->level = level;
    }

    bool operator==(const state_t &other) const {
        return this->word == other.word;
    }
};

namespace std {
    template<> struct hash<state_t> {
    public:
        size_t operator()(const state_t& s) const {
            return str_hash(s.word);
        }
    private:
        std::hash<std::string> str_hash;
    };
}


class Solution {
public:
    vector<vector<string> > findLadders(const string& start,
        const string& end, const unordered_set<string> &dict) {
        queue<state_t> q;
        unordered_set<state_t> visited; // 判重
        unordered_map<state_t, vector<state_t> > father; // DAG

        auto state_is_valid = [&](const state_t& s) {
            return dict.find(s.word) != dict.end() || s.word == end;
        };
        auto state_is_target = [&](const state_t &s) {return s.word == end; };
        auto state_extend = [&](const state_t &s) {
            unordered_set<state_t> result;

            for (size_t i = 0; i < s.word.size(); ++i) {
                state_t new_state(s.word, s.level + 1);
                for (char c = 'a'; c <= 'z'; c++) {
                    // 防止同字母替换
                    if (c == new_state.word[i]) continue;

                    swap(c, new_state.word[i]);

                    if (state_is_valid(new_state)) {
                        auto visited_iter = visited.find(new_state);

                        if (visited_iter != visited.end()) {
                            if (visited_iter->level < new_state.level) {
                                // do nothing
                            } else if (visited_iter->level == new_state.level) {
                                result.insert(new_state);
                            } else { // not possible
                                throw std::logic_error("not possible to get here");
                            }
                        } else {
                            result.insert(new_state);
                        }
                    }
                    swap(c, new_state.word[i]); // 恢复该单词
                }
            }

            return result;
        };

        vector<vector<string>> result;
        state_t start_state(start, 0);
        q.push(start_state);
        visited.insert(start_state);
        while (!q.empty()) {
            // 千万不能用 const auto&,pop() 会删除元素,
            // 引用就变成了悬空引用
            const auto state = q.front();
            q.pop();

            // 如果当前路径长度已经超过当前最短路径长度,
            // 可以中止对该路径的处理,因为我们要找的是最短路径
            if (!result.empty() && state.level + 1 > result[0].size()) break;

            if (state_is_target(state)) {
                vector<string> path;
                gen_path(father, start_state, state, path, result);
                continue;
            }

            // 扩展节点
            const auto& new_states = state_extend(state);
            for (const auto& new_state : new_states) {
                if (visited.find(new_state) == visited.end()) {
                    q.push(new_state);
                }
                visited.insert(new_state);
                father[new_state].push_back(state);
            }
        }

        return result;
    }
private:
    void gen_path(unordered_map<state_t, vector<state_t> > &father,
        const state_t &start, const state_t &state, vector<string> &path,
        vector<vector<string> > &result) {
        path.push_back(state.word);
        if (state == start) {
            if (!result.empty()) {
                if (path.size() < result[0].size()) {
                    result.clear();
                    result.push_back(path);
                    reverse(result.back().begin(), result.back().end());
                } else if (path.size() == result[0].size()) {
                    result.push_back(path);
                    reverse(result.back().begin(), result.back().end());
                } else { // not possible
                    throw std::logic_error("not possible to get here ");
                }
            } else {
                result.push_back(path);
                reverse(result.back().begin(), result.back().end());
            }

        } else {
            for (const auto& f : father[state]) {
                gen_path(father, start, f, path, result);
            }
        }
        path.pop_back();
    }
};
\end{Code}


\subsubsection{双队列}

\begin{Code}
//LeetCode, Word Ladder II
// 时间复杂度O(n),空间复杂度O(n)
class Solution {
public:
    vector<vector<string> > findLadders(const string& start,
            const string& end, const unordered_set<string> &dict) {
        // 当前层,下一层,用unordered_set是为了去重,例如两个父节点指向
        // 同一个子节点,如果用vector, 子节点就会在next里出现两次,其实此
        // 时 father 已经记录了两个父节点,next里重复出现两次是没必要的
        unordered_set<string> current, next;
        unordered_set<string> visited; // 判重
        unordered_map<string, vector<string> > father; // DAG

        int level = -1;  // 层次

        auto state_is_valid = [&](const string& s) {
            return dict.find(s) != dict.end() || s == end;
        };
        auto state_is_target = [&](const string &s) {return s == end;};
        auto state_extend = [&](const string &s) {
            unordered_set<string> result;

            for (size_t i = 0; i < s.size(); ++i) {
                string new_word(s);
                for (char c = 'a'; c <= 'z'; c++) {
                    // 防止同字母替换
                    if (c == new_word[i]) continue;

                    swap(c, new_word[i]);

                    if (state_is_valid(new_word) &&
                            visited.find(new_word) == visited.end()) {
                        result.insert(new_word);
                    }
                    swap(c, new_word[i]); // 恢复该单词
                }
            }

            return result;
        };

        vector<vector<string> > result;
        current.insert(start);
        while (!current.empty()) {
            ++ level;
            // 如果当前路径长度已经超过当前最短路径长度,可以中止对该路径的
            // 处理,因为我们要找的是最短路径
            if (!result.empty() && level+1 > result[0].size()) break;

            // 1. 延迟加入visited, 这样才能允许两个父节点指向同一个子节点
            // 2. 一股脑current 全部加入visited, 是防止本层前一个节点扩展
            // 节点时,指向了本层后面尚未处理的节点,这条路径必然不是最短的
            for (const auto& state : current)
                visited.insert(state);
            for (const auto& state : current) {
                if (state_is_target(state)) {
                    vector<string> path;
                    gen_path(father, path, start, state, result);
                    continue;
                }

                const auto new_states = state_extend(state);
                for (const auto& new_state : new_states) {
                    next.insert(new_state);
                    father[new_state].push_back(state);
                }
            }

            current.clear();
            swap(current, next);
        }

        return result;
    }
private:
    void gen_path(unordered_map<string, vector<string> > &father,
            vector<string> &path, const string &start, const string &word,
            vector<vector<string> > &result) {
        path.push_back(word);
        if (word == start) {
            if (!result.empty()) {
                if (path.size() < result[0].size()) {
                    result.clear();
                    result.push_back(path);
                } else if(path.size() == result[0].size()) {
                    result.push_back(path);
                } else {
                    // not possible
                    throw std::logic_error("not possible to get here");
                }
            } else {
                result.push_back(path);
            }
            reverse(result.back().begin(), result.back().end());
        } else {
            for (const auto& f : father[word]) {
                gen_path(father, path, start, f, result);
            }
        }
        path.pop_back();
    }
};
\end{Code}


\subsubsection{图的广搜}

本题还可以看做是图上的广搜。给定了字典 \fn{dict},可以基于它画出一个无向图,表示单词之间可以互相转换。本题的本质就是已知起点和终点,在图上找出所有最短路径。

\begin{Code}
//LeetCode, Word Ladder II
// 时间复杂度O(n),空间复杂度O(n)
class Solution {
public:
    vector<vector<string> > findLadders(const string& start,
            const string &end, const unordered_set<string> &dict) {
        const auto& g = build_graph(dict);
        vector<state_t*> pool;
        queue<state_t*> q; // 未处理的节点
        // value 是所在层次
        unordered_map<string, int> visited;

        auto state_is_target = [&](const state_t *s) {return s->word == end; };

        vector<vector<string>> result;
        q.push(create_state(nullptr, start, 0, pool));
        while (!q.empty()) {
            state_t* state = q.front();
            q.pop();

            // 如果当前路径长度已经超过当前最短路径长度,
            // 可以中止对该路径的处理,因为我们要找的是最短路径
            if (!result.empty() && state->level+1 > result[0].size()) break;

            if (state_is_target(state)) {
                const auto& path = gen_path(state);
                if (result.empty()) {
                    result.push_back(path);
                } else {
                    if (path.size() < result[0].size()) {
                        result.clear();
                        result.push_back(path);
                    } else if (path.size() == result[0].size()) {
                        result.push_back(path);
                    } else {
                        // not possible
                        throw std::logic_error("not possible to get here");
                    }
                }
                continue;
            }
            visited[state->word] = state->level;

            // 扩展节点
            auto iter = g.find(state->word);
            if (iter == g.end()) continue;

            for (const auto& neighbor : iter->second) {
                auto visited_iter = visited.find(neighbor);

                if (visited_iter != visited.end() && 
                    visited_iter->second < state->level + 1) {
                    continue;
                }

                q.push(create_state(state, neighbor, state->level + 1, pool));
            }
        }

        // release all states
        for (auto state : pool) {
            delete state;
        }
        return result;
    }

private:
    struct state_t {
        state_t* father;
        string word;
        int level; // 所在层次,从0开始编号

        state_t(state_t* father_, const string& word_, int level_) :
            father(father_), word(word_), level(level_) {}
    };

    state_t* create_state(state_t* parent, const string& value,
            int length, vector<state_t*>& pool) {
        state_t* node = new state_t(parent, value, length);
        pool.push_back(node);

        return node;
    }
    vector<string> gen_path(const state_t* node) {
        vector<string> path;

        while(node != nullptr) {
            path.push_back(node->word);
            node = node->father;
        }

        reverse(path.begin(), path.end());
        return path;
    }

    unordered_map<string, unordered_set<string> > build_graph(
            const unordered_set<string>& dict) {
        unordered_map<string, unordered_set<string> > adjacency_list;

        for (const auto& word : dict) {
            for (size_t i = 0; i < word.size(); ++i) {
                string new_word(word);
                for (char c = 'a'; c <= 'z'; c++) {
                    // 防止同字母替换
                    if (c == new_word[i]) continue;

                    swap(c, new_word[i]);

                    if ((dict.find(new_word) != dict.end())) {
                        auto iter = adjacency_list.find(word);
                        if (iter != adjacency_list.end()) {
                            iter->second.insert(new_word);
                        } else {
                            adjacency_list.insert(pair<string,
                                unordered_set<string>>(word, unordered_set<string>()));
                            adjacency_list[word].insert(new_word);
                        }
                    }
                    swap(c, new_word[i]); // 恢复该单词
                }
            }
        }
        return adjacency_list;
    }
};
\end{Code}


\subsubsection{相关题目}

\begindot
\item Word Ladder,见 \S \ref{sec:word-ladder}
\myenddot


\section{Surrounded Regions} %%%%%%%%%%%%%%%%%%%%%%%%%%%%%%
\label{sec:surrounded-regions}


\subsubsection{描述}
Given a 2D board containing \fn{'X'} and \fn{'O'}, capture all regions surrounded by \fn{'X'}.

A region is captured by flipping all \fn{'O'}s into \fn{'X'}s in that surrounded region .

For example,
\begin{Code}
X X X X
X O O X
X X O X
X O X X
\end{Code}

After running your function, the board should be:
\begin{Code}
X X X X
X X X X
X X X X
X O X X
\end{Code}


\subsubsection{分析}
广搜。从上下左右四个边界往里走,凡是能碰到的\fn{'O'},都是跟边界接壤的,应该保留。


\subsubsection{代码}
\begin{Code}
// LeetCode, Surrounded Regions
// BFS,时间复杂度O(n),空间复杂度O(n)
class Solution {
public:
    void solve(vector<vector<char>> &board) {
        if (board.empty()) return;

        const int m = board.size();
        const int n = board[0].size();
        for (int i = 0; i < n; i++) {
            bfs(board, 0, i);
            bfs(board, m - 1, i);
        }
        for (int j = 1; j < m - 1; j++) {
            bfs(board, j, 0);
            bfs(board, j, n - 1);
        }
        for (int i = 0; i < m; i++)
            for (int j = 0; j < n; j++)
                if (board[i][j] == 'O')
                    board[i][j] = 'X';
                else if (board[i][j] == '+')
                    board[i][j] = 'O';
    }
private:
    void bfs(vector<vector<char>> &board, int i, int j) {
        typedef pair<int, int> state_t;
        queue<state_t> q;
        const int m = board.size();
        const int n = board[0].size();

        auto state_is_valid = [&](const state_t &s) {
            const int x = s.first;
            const int y = s.second;
            if (x < 0 || x >= m || y < 0 || y >= n || board[x][y] != 'O')
                return false;
            return true;
        };

        auto state_extend = [&](const state_t &s) {
            vector<state_t> result;
            const int x = s.first;
            const int y = s.second;
            // 上下左右
            const state_t new_states[4] = {{x-1,y}, {x+1,y},
                    {x,y-1}, {x,y+1}};
            for (int k = 0; k < 4;  ++k) {
                if (state_is_valid(new_states[k])) {
                    // 既有标记功能又有去重功能
                    board[new_states[k].first][new_states[k].second] = '+';
                    result.push_back(new_states[k]);
                }
            }

            return result;
        };

        state_t start = { i, j };
        if (state_is_valid(start)) {
            board[i][j] = '+';
            q.push(start);
        }
        while (!q.empty()) {
            auto cur = q.front();
            q.pop();
            auto new_states = state_extend(cur);
            for (auto s : new_states) q.push(s);
        }
    }
};
\end{Code}


\subsubsection{相关题目}

\begindot
\item 无
\myenddot


\section{小结} %%%%%%%%%%%%%%%%%%%%%%%%%%%%%%
\label{sec:bfs-template}


\subsection{适用场景}

\textbf{输入数据}:没什么特征,不像深搜,需要有“递归”的性质。如果是树或者图,概率更大。

\textbf{状态转换图}:树或者DAG图。

\textbf{求解目标}:求最短。


\subsection{思考的步骤}
\begin{enumerate}
\item 是求路径长度,还是路径本身(或动作序列)?
    \begin{enumerate}
    \item 如果是求路径长度,则状态里面要存路径长度(或双队列+一个全局变量)
    \item 如果是求路径本身或动作序列
        \begin{enumerate}
            \item 要用一棵树存储宽搜过程中的路径
            \item 是否可以预估状态个数的上限?能够预估状态总数,则开一个大数组,用树的双亲表示法;如果不能预估状态总数,则要使用一棵通用的树。这一步也是第4步的必要不充分条件。
        \end{enumerate}
    \end{enumerate}

\item 如何表示状态?即一个状态需要存储哪些些必要的数据,才能够完整提供如何扩展到下一步状态的所有信息。一般记录当前位置或整体局面。

\item 如何扩展状态?这一步跟第2步相关。状态里记录的数据不同,扩展方法就不同。对于固定不变的数据结构(一般题目直接给出,作为输入数据),如二叉树,图等,扩展方法很简单,直接往下一层走,对于隐式图,要先在第1步里想清楚状态所带的数据,想清楚了这点,那如何扩展就很简单了。

\item 如何判断重复?如果状态转换图是一颗树,则永远不会出现回路,不需要判重;如果状态转换图是一个图(这时候是一个图上的BFS),则需要判重。
    \begin{enumerate}
    \item 如果是求最短路径长度或一条路径,则只需要让“点”(即状态)不重复出现,即可保证不出现回路
    \item 如果是求所有路径,注意此时,状态转换图是DAG,即允许两个父节点指向同一个子节点。具体实现时,每个节点要\textbf{“延迟”}加入到已访问集合\fn{visited},要等一层全部访问完后,再加入到\fn{visited}集合。
    \item 具体实现
        \begin{enumerate}
        \item 状态是否存在完美哈希方案?即将状态一一映射到整数,互相之间不会冲突。
        \item 如果不存在,则需要使用通用的哈希表(自己实现或用标准库,例如\fn{unordered_set})来判重;自己实现哈希表的话,如果能够预估状态个数的上限,则可以开两个数组,head和next,表示哈希表,参考第 \S \ref{subsec:eightDigits}节方案2。
        \item 如果存在,则可以开一个大布尔数组,来判重,且此时可以精确计算出状态总数,而不仅仅是预估上限。
        \end{enumerate}
    \end{enumerate}

\item 目标状态是否已知?如果题目已经给出了目标状态,可以带来很大便利,这时候可以从起始状态出发,正向广搜;也可以从目标状态出发,逆向广搜;也可以同时出发,双向广搜。
\end{enumerate}


\subsection{代码模板}
广搜需要一个队列,用于一层一层扩展,一个hashset,用于判重,一棵树(只求长度时不需要),用于存储整棵树。

对于队列,可以用\fn{queue},也可以把\fn{vector}当做队列使用。当求长度时,有两种做法:
\begin{enumerate}
\item 只用一个队列,但在状态结构体\fn{state_t}里增加一个整数字段\fn{level},表示当前所在的层次,当碰到目标状态,直接输出\fn{level}即可。这个方案,可以很容易的变成A*算法,把\fn{queue}替换为\fn{priority_queue}即可。
\item 用两个队列,\fn{current, next},分别表示当前层次和下一层,另设一个全局整数\fn{level},表示层数(也即路径长度),当碰到目标状态,输出\fn{level}即可。这个方案,状态里可以不存路径长度,只需全局设置一个整数\fn{level},比较节省内存;
\end{enumerate}

对于hashset,如果有完美哈希方案,用布尔数组(\fn{bool visited[STATE_MAX]}或\fn{vector<bool> visited(STATE_MAX, false)})来表示;如果没有,可以用STL里的\fn{set}或\fn{unordered_set}。

对于树,如果用STL,可以用\fn{unordered_map<state_t, state_t > father}表示一颗树,代码非常简洁。如果能够预估状态总数的上限(设为STATE_MAX),可以用数组(\fn{state_t nodes[STATE_MAX]}),即树的双亲表示法来表示树,效率更高,当然,需要写更多代码。


\subsubsection{如何表示状态}

\begin{Codex}[label=bfs_common.h]
/** 状态 */
struct state_t {
    int data1;  /** 状态的数据,可以有多个字段. */
    int data2;  /** 状态的数据,可以有多个字段. */
    // dataN;   /** 其他字段 */
    int action; /** 由父状态移动到本状态的动作,求动作序列时需要. */
    int level;  /** 所在的层次(从0开始),也即路径长度-1,求路径长度时需要;
                    不过,采用双队列时不需要本字段,只需全局设一个整数 */
    bool operator==(const state_t &other) const {
        return true;  // 根据具体问题实现
    }
};

// 定义hash函数

// 方法1:模板特化,当hash函数只需要状态本身,不需要其他数据时,用这个方法比较简洁
namespace std {
template<> struct hash<state_t> {
    size_t operator()(const state_t & x) const {
        return 0; // 根据具体问题实现
    }
};
}

// 方法2:函数对象,如果hash函数需要运行时数据,则用这种方法
class Hasher {
public:
    Hasher(int _m) : m(_m) {};
    size_t operator()(const state_t &s) const {
        return 0; // 根据具体问题实现
    }
private:
    int m; // 存放外面传入的数据
};

/**
 * @brief 反向生成路径,求一条路径.
 * @param[in] father 树
 * @param[in] target 目标节点
 * @return 从起点到target的路径
 */
vector<state_t> gen_path(const unordered_map<state_t, state_t> &father,
        const state_t &target) {
    vector<state_t> path;
    path.push_back(target);

    for (state_t cur = target; father.find(cur) != father.end(); 
            cur = father.at(cur))
        path.push_back(cur);

    reverse(path.begin(), path.end());

    return path;
}

/**
 * 反向生成路径,求所有路径.
 * @param[in] father 存放了所有路径的树
 * @param[in] start 起点
 * @param[in] state 终点
 * @return 从起点到终点的所有路径
 */
void gen_path(unordered_map<state_t, vector<state_t> > &father,
        const string &start, const state_t& state, vector<state_t> &path,
        vector<vector<state_t> > &result) {
    path.push_back(state);
    if (state == start) {
        if (!result.empty()) {
            if (path.size() < result[0].size()) {
                result.clear();
                result.push_back(path);
            } else if(path.size() == result[0].size()) {
                result.push_back(path);
            } else {
                // not possible
                throw std::logic_error("not possible to get here");
            }
        } else {
            result.push_back(path);
        }
        reverse(result.back().begin(), result.back().end());
    } else {
        for (const auto& f : father[state]) {
            gen_path(father, start, f, path, result);
        }
    }
    path.pop_back();
}
\end{Codex}


\subsubsection{求最短路径长度或一条路径}

\textbf{单队列的写法}

\begin{Codex}[label=bfs_template.cpp]
#include "bfs_common.h"

/**
 * @brief 广搜,只用一个队列.
 * @param[in] start 起点
 * @param[in] data 输入数据
 * @return 从起点到目标状态的一条最短路径
 */
vector<state_t> bfs(state_t &start, const vector<vector<int>> &grid) {
    queue<state_t> q; // 队列
    unordered_set<state_t> visited; // 判重
    unordered_map<state_t, state_t> father; // 树,求路径本身时才需要

    // 判断状态是否合法
    auto state_is_valid = [&](const state_t &s) { /*...*/ };

    // 判断当前状态是否为所求目标
    auto state_is_target = [&](const state_t &s) { /*...*/ };

    // 扩展当前状态
    auto state_extend = [&](const state_t &s) {
        unordered_set<state_t> result;
        for (/*...*/) {
            const state_t new_state = /*...*/;
            if (state_is_valid(new_state) && 
                    visited.find(new_state) != visited.end()) {
                result.insert(new_state);
            }
        }
        return result;
    };

    assert (start.level == 0);
    q.push(start);
    while (!q.empty()) {
        // 千万不能用 const auto&,pop() 会删除元素,
        // 引用就变成了悬空引用
        const state_t state = q.front();
        q.pop();
        visited.insert(state);

        // 访问节点
        if (state_is_target(state)) {
            return return gen_path(father, target); // 求一条路径
            // return state.level + 1; // 求路径长度
        }

        // 扩展节点
        vector<state_t> new_states = state_extend(state);
        for (const auto& new_state : new_states) {
            q.push(new_state);
            father[new_state] = state;  // 求一条路径
            // visited.insert(state); // 优化:可以提前加入 visited 集合,
            // 从而缩小状态扩展。这时 q 的含义略有变化,里面存放的是处理了一半
            // 的节点:已经加入了visited,但还没有扩展。别忘记 while循环开始
            // 前,要加一行代码, visited.insert(start)
        }
    }

    return vector<state_t>();
    //return 0;
}
\end{Codex}


\textbf{双队列的写法}
\begin{Codex}[label=bfs_template1.cpp]
#include "bfs_common.h"

/**
 * @brief 广搜,使用两个队列.
 * @param[in] start 起点
 * @param[in] data 输入数据
 * @return 从起点到目标状态的一条最短路径
 */
vector<state_t> bfs(const state_t &start, const type& data) {
    queue<state_t> next, current; // 当前层,下一层
    unordered_set<state_t> visited; // 判重
    unordered_map<state_t, state_t> father; // 树,求路径本身时才需要

    int level = -1;  // 层次

    // 判断状态是否合法
    auto state_is_valid = [&](const state_t &s) { /*...*/ };

    // 判断当前状态是否为所求目标
    auto state_is_target = [&](const state_t &s) { /*...*/ };

    // 扩展当前状态
    auto state_extend = [&](const state_t &s) {
        unordered_set<state_t> result;
        for (/*...*/) {
            const state_t new_state = /*...*/;
            if (state_is_valid(new_state) && 
                    visited.find(new_state) != visited.end()) {
                result.insert(new_state);
            }
        }
        return result;
    };

    current.push(start);
    while (!current.empty()) {
        ++level;
        while (!current.empty()) {
            // 千万不能用 const auto&,pop() 会删除元素,
            // 引用就变成了悬空引用
            const auto state = current.front();
            current.pop();
            visited.insert(state);

            if (state_is_target(state)) {
                return return gen_path(father, state); // 求一条路径
                // return state.level + 1; // 求路径长度
            }

            const auto& new_states = state_extend(state);
            for (const auto& new_state : new_states) {
                next.push(new_state);
                father[new_state] = state;
                // visited.insert(state); // 优化:可以提前加入 visited 集合,
                // 从而缩小状态扩展。这时 current 的含义略有变化,里面存放的是处
                // 理了一半的节点:已经加入了visited,但还没有扩展。别忘记 while
                // 循环开始前,要加一行代码, visited.insert(start)
            }
        }
        swap(next, current); //!!! 交换两个队列
    }

    return vector<state_t>();
    // return 0;
}
\end{Codex}


\subsubsection{求所有路径}

\textbf{单队列}

\begin{Codex}[label=bfs_template.cpp]
/**
 * @brief 广搜,使用一个队列.
 * @param[in] start 起点
 * @param[in] data 输入数据
 * @return 从起点到目标状态的所有最短路径
 */
vector<vector<state_t> > bfs(const state_t &start, const type& data) {
    queue<state_t> q;
    unordered_set<state_t> visited; // 判重
    unordered_map<state_t, vector<state_t> > father; // DAG

    auto state_is_valid = [&](const state_t& s) { /*...*/ };
    auto state_is_target = [&](const state_t &s) { /*...*/ };
    auto state_extend = [&](const state_t &s) {
        unordered_set<state_t> result;
        for (/*...*/) {
            const state_t new_state = /*...*/;
            if (state_is_valid(new_state)) {
                auto visited_iter = visited.find(new_state);

                if (visited_iter != visited.end()) {
                    if (visited_iter->level < new_state.level) {
                        // do nothing
                    } else if (visited_iter->level == new_state.level) {
                        result.insert(new_state);
                    } else { // not possible
                        throw std::logic_error("not possible to get here");
                    }
                } else {
                    result.insert(new_state);
                }
            }
        }

        return result;
    };

    vector<vector<string>> result;
    state_t start_state(start, 0);
    q.push(start_state);
    visited.insert(start_state);
    while (!q.empty()) {
        // 千万不能用 const auto&,pop() 会删除元素,
        // 引用就变成了悬空引用
        const auto state = q.front();
        q.pop();

        // 如果当前路径长度已经超过当前最短路径长度,
        // 可以中止对该路径的处理,因为我们要找的是最短路径
        if (!result.empty() && state.level + 1 > result[0].size()) break;

        if (state_is_target(state)) {
            vector<string> path;
            gen_path(father, start_state, state, path, result);
            continue;
        }

        // 扩展节点
        const auto& new_states = state_extend(state);
        for (const auto& new_state : new_states) {
            if (visited.find(new_state) == visited.end()) {
                q.push(new_state);
            }
            visited.insert(new_state);
            father[new_state].push_back(state);
        }
    }

    return result;
}
\end{Codex}


\textbf{双队列的写法}

\begin{Codex}[label=bfs_template.cpp]
#include "bfs_common.h"

/**
 * @brief 广搜,使用两个队列.
 * @param[in] start 起点
 * @param[in] data 输入数据
 * @return 从起点到目标状态的所有最短路径
 */
vector<vector<state_t> > bfs(const state_t &start, const type& data) {
    // 当前层,下一层,用unordered_set是为了去重,例如两个父节点指向
    // 同一个子节点,如果用vector, 子节点就会在next里出现两次,其实此
    // 时 father 已经记录了两个父节点,next里重复出现两次是没必要的
    unordered_set<string> current, next;
    unordered_set<state_t> visited; // 判重
    unordered_map<state_t, vector<state_t> > father; // DAG

    int level = -1;  // 层次

    // 判断状态是否合法
    auto state_is_valid = [&](const state_t &s) { /*...*/ };

    // 判断当前状态是否为所求目标
    auto state_is_target = [&](const state_t &s) { /*...*/ };

    // 扩展当前状态
    auto state_extend = [&](const state_t &s) {
        unordered_set<state_t> result;
        for (/*...*/) {
            const state_t new_state = /*...*/;
            if (state_is_valid(new_state) && 
                    visited.find(new_state) != visited.end()) {
                result.insert(new_state);
            }
        }
        return result;
    };

    vector<vector<state_t> > result;
    current.insert(start);
    while (!current.empty()) {
        ++ level;
        // 如果当前路径长度已经超过当前最短路径长度,可以中止对该路径的
        // 处理,因为我们要找的是最短路径
        if (!result.empty() && level+1 > result[0].size()) break;

        // 1. 延迟加入visited, 这样才能允许两个父节点指向同一个子节点
        // 2. 一股脑current 全部加入visited, 是防止本层前一个节点扩展
        // 节点时,指向了本层后面尚未处理的节点,这条路径必然不是最短的
        for (const auto& state : current)
            visited.insert(state);
        for (const auto& state : current) {
            if (state_is_target(state)) {
                vector<string> path;
                gen_path(father, path, start, state, result);
                continue;
            }

            const auto new_states = state_extend(state);
            for (const auto& new_state : new_states) {
                next.insert(new_state);
                father[new_state].push_back(state);
            }
        }

        current.clear();
        swap(current, next);
    }

    return result;
}
\end{Codex}

