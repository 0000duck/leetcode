\chapter{线性表}
这类题目考察线性表的操作,例如,数组,单链表,双向链表等。


\section{数组} %%%%%%%%%%%%%%%%%%%%%%%%%%%%%%


\subsection{Remove Duplicates from Sorted Array}
\label{sec:remove-duplicates-from-sorted-array}


\subsubsection{描述}
Given a sorted array, remove the duplicates in place such that each element appear only once and return the new length.

Do not allocate extra space for another array, you must do this in place with constant memory.

For example, Given input array \code{A = [1,1,2]},

Your function should return length = 2, and A is now \code{[1,2]}.


\subsubsection{分析}



\subsubsection{代码}
\begin{Code}
// LeetCode, Remove Duplicates from Sorted Array
class Solution {
public:
    int removeDuplicates(int A[], int n) {
        if (n == 0) return 0;

        int index = 0;
        for (int i = 1; i < n; i++) {
            if (A[index] != A[i])
                A[++index] = A[i];
        }
        return index + 1;
    }
};
\end{Code}


\begin{Code}
// LeetCode, Remove Duplicates from Sorted Array,使用STL
class Solution {
public:
    int removeDuplicates(int A[], int n) {
        return removeDuplicates(A, A + n, A) - A;
    }

    template<typename InIt, typename OutIt>
    OutIt removeDuplicates(InIt first, InIt last, OutIt output) {
        while (first != last) {
            *output++ = *first;
            first = std::find_if(first, last,
                    std::bind1st(std::not_equal_to<int>(), *first));
        }

        return output;
    }
};
\end{Code}


\subsubsection{相关题目}

\begindot
\item Remove Duplicates from Sorted Array II,见 \S \ref{sec:remove-duplicates-from-sorted-array-ii}
\myenddot


\subsection{Remove Duplicates from Sorted Array II}
\label{sec:remove-duplicates-from-sorted-array-ii}


\subsubsection{描述}
Follow up for "Remove Duplicates":
What if duplicates are allowed at most twice?

For example,
Given sorted array \code{A = [1,1,1,2,2,3]},

Your function should return length = 5, and A is now \code{[1,1,2,2,3]}


\subsubsection{分析}
加一个变量记录一下元素出现的次数即可。这题因为是已经排序的数组,所以一个变量即可解决。如果是没有排序的数组,则需要引入一个hashmap来记录出现次数。


\subsubsection{代码}
\begin{Code}
// LeetCode, Remove Duplicates from Sorted Array II
class Solution {
public:
    int removeDuplicates(int A[], int n) {
        if (n == 0) return 0;

        int occur = 1;
        int index = 0;
        for (int i = 1; i < n; i++) {
            if (A[index] == A[i]) {
                if (occur < 2) {
                    A[++index] = A[i];
                    occur++;
                }
            } else {
                A[++index] = A[i];
                occur = 1;
            }
        }
        return index + 1;
    }
};
\end{Code}


\subsubsection{相关题目}

\begindot
\item Remove Duplicates from Sorted Array,见 \S \ref{sec:remove-duplicates-from-sorted-array}
\myenddot


\subsection{Search in Rotated Sorted Array}
\label{sec:search-in-rotated-sorted-array}


\subsubsection{描述}
Suppose a sorted array is rotated at some pivot unknown to you beforehand.

(i.e., \code{0 1 2 4 5 6 7} might become \code{4 5 6 7 0 1 2}).

You are given a target value to search. If found in the array return its index, otherwise return -1.

You may assume no duplicate exists in the array.


\subsubsection{分析}
二分查找,难度主要在于左右边界的确定。


\subsubsection{代码}
\begin{Code}
// LeetCode, Search in Rotated Sorted Array
class Solution {
public:
    int search(int A[], int n, int target) {
        int first = 0, last = n;
        while (first != last) {
            const int mid = (first + last) / 2;
            if (A[mid] == target)
                return mid;
            if (A[first] <= A[mid]) {
                if (A[first] <= target && target < A[mid])
                    last = mid;
                else
                    first = mid + 1;
            } else {
                if (A[mid] < target && target <= A[last-1])
                    first = mid + 1;
                else
                    last = mid;
            }
        }
        return -1;
    }
};
\end{Code}


\subsubsection{相关题目}

\begindot
\item Search in Rotated Sorted Array II,见 \S \ref{sec:search-in-rotated-sorted-array-ii}
\myenddot


\subsection{Search in Rotated Sorted Array II}
\label{sec:search-in-rotated-sorted-array-ii}


\subsubsection{描述}
Follow up for "Search in Rotated Sorted Array": What if \emph{duplicates} are allowed?

Would this affect the run-time complexity? How and why?

Write a function to determine if a given target is in the array.


\subsubsection{分析}
允许重复元素,则上一题中如果\fn{A[m]>=A[l]},那么\fn{[l,m]}为递增序列的假设就不能成立了,比如\code{[1,3,1,1,1]}。

如果\fn{A[m]>=A[l]}不能确定递增,那就把它拆分成两个条件:
\begindot
\item 若\fn{A[m]>A[l]},则区间\fn{[l,m]}一定递增
\item 若\fn{A[m]==A[l]} 确定不了,那就\fn{l++},往下看一步即可。
\myenddot

\subsubsection{代码}
\begin{Code}
// LeetCode, Search in Rotated Sorted Array II
class Solution {
public:
    bool search(int A[], int n, int target) {
        int first = 0, last = n;
        while (first != last) {
            const int mid = (first + last) / 2;
            if (A[mid] == target)
                return true;
            if (A[first] < A[mid]) {
                if (A[first] <= target && target < A[mid])
                    last = mid;
                else
                    first = mid + 1;
            } else if (A[first] > A[mid]) {
                if (A[mid] <= target && target <= A[last-1])
                    first = mid + 1;
                else
                    last = mid;
            } else
                //skip duplicate one, A[start] == A[mid]
                first++;
        }
        return false;
    }
};
\end{Code}


\subsubsection{相关题目}

\begindot
\item Search in Rotated Sorted Array,见 \S \ref{sec:search-in-rotated-sorted-array}
\myenddot


\subsection{Median of Two Sorted Arrays}
\label{sec:median-of-two-sorted-arrays}


\subsubsection{描述}
There are two sorted arrays A and B of size m and n respectively. Find the median of the two sorted arrays. The overall run time complexity should be $O(\log (m+n))$.


\subsubsection{分析}
这是一道非常经典的题。这题更通用的形式是,给定两个已经排序好的数组,找到两者所有元素中第$k$大的元素。

$O(m+n)$的解法比较直观,直接merge两个数组,然后求第$k$大的元素。

不过我们仅仅需要第$k$大的元素,是不需要“排序”这么复杂的操作的。可以用一个计数器,记录当前已经找到第$m$大的元素了。同时我们使用两个指针\fn{pA}和\fn{pB},分别指向A和B数组的第一个元素,使用类似于merge sort的原理,如果数组A当前元素小,那么\fn{pA++},同时\fn{m++};如果数组B当前元素小,那么\fn{pB++},同时\fn{m++}。最终当$m$等于$k$的时候,就得到了我们的答案,$O(k)$时间,$O(1)$空间。但是,当$k$很接近$m+n$的时候,这个方法还是$O(m+n)$的。

有没有更好的方案呢?我们可以考虑从$k$入手。如果我们每次都能够删除一个一定在第$k$大元素之前的元素,那么我们需要进行$k$次。但是如果每次我们都删除一半呢?由于A和B都是有序的,我们应该充分利用这里面的信息,类似于二分查找,也是充分利用了“有序”。

假设A和B的元素个数都大于$k/2$,我们将A的第$k/2$个元素(即\fn{A[k/2-1]})和B的第$k/2$个元素(即\fn{B[k/2-1]})进行比较,有以下三种情况(为了简化这里先假设$k$为偶数,所得到的结论对于$k$是奇数也是成立的):
\begindot
\item \fn{A[k/2-1] == B[k/2-1]}
\item \fn{A[k/2-1] > B[k/2-1]}
\item \fn{A[k/2-1] < B[k/2-1]}
\myenddot

如果\fn{A[k/2-1] < B[k/2-1]},意味着\fn{A[0]}到\fn{A[k/2-1}的肯定在$A \cup B$的top k元素的范围内,换句话说,\fn{A[k/2-1}不可能大于$A \cup B$的第$k$大元素。留给读者证明。

因此,我们可以放心的删除A数组的这$k/2$个元素。同理,当\fn{A[k/2-1] > B[k/2-1]}时,可以删除B数组的$k/2$个元素。

当\fn{A[k/2-1] == B[k/2-1]}时,说明找到了第$k$大的元素,直接返回\fn{A[k/2-1]}或\fn{B[k/2-1]}即可。

因此,我们可以写一个递归函数。那么函数什么时候应该终止呢?
\begindot
\item 当A或B是空时,直接返回\fn{B[k-1]}或\fn{A[k-1]};
\item 当\fn{k=1}是,返回\fn{min(A[0], B[0])};
\item 当\fn{A[k/2-1] == B[k/2-1]}时,返回\fn{A[k/2-1]}或\fn{B[k/2-1]}
\myenddot


\subsubsection{代码}
\begin{Code}
// LeetCode, Median of Two Sorted Arrays
class Solution {
public:
    double findMedianSortedArrays(int A[], int m, int B[], int n) {
        int total = m + n;
        if (total & 0x1)
            return find_kth(A, m, B, n, total / 2 + 1);
        else
            return (find_kth(A, m, B, n, total / 2)
                    + find_kth(A, m, B, n, total / 2 + 1)) / 2;
    }
private:
    static double find_kth(int A[], int m, int B[], int n, int k) {
        //always assume that m is equal or smaller than n
        if (m > n) return find_kth(B, n, A, m, k);
        if (m == 0) return B[k - 1];
        if (k == 1) return min(A[0], B[0]);

        //divide k into two parts
        int pa = min(k / 2, m), pb = k - pa;
        if (A[pa - 1] < B[pb - 1])
            return find_kth(A + pa, m - pa, B, n, k - pa);
        else if (A[pa - 1] > B[pb - 1])
            return find_kth(A, m, B + pb, n - pb, k - pb);
        else
            return A[pa - 1];
    }
};
\end{Code}


\subsubsection{相关题目}

\begindot
\item 无
\myenddot


\section{单链表} %%%%%%%%%%%%%%%%%%%%%%%%%%%%%%

单链表节点的定义如下:
\begin{Code}
// 单链表节点
struct ListNode {
    int val;
    ListNode *next;
    ListNode(int x) : val(x), next(nullptr) { }
};
\end{Code}


\subsection{Add Two Numbers}
\label{sec:add-two-numbers}


\subsubsection{描述}
You are given two linked lists representing two non-negative numbers. The digits are stored in reverse order and each of their nodes contain a single digit. Add the two numbers and return it as a linked list.

Input: {\small \fontspec{Latin Modern Mono} (2 -> 4 -> 3) + (5 -> 6 -> 4)}

Output: {\small \fontspec{Latin Modern Mono} 7 -> 0 -> 8}


\subsubsection{分析}
跟Add Binary(见 \S \ref{sec:add-binary})很类似


\subsubsection{代码}
\begin{Code}
//LeetCode, Add Two Numbers
//跟Add Binary 很类似
class Solution {
public:
    ListNode *addTwoNumbers(ListNode *l1, ListNode *l2) {
        ListNode head(-1); // 头节点
        int carry = 0;
        ListNode *prev = &head;
        for (ListNode *pa = l1, *pb = l2;
             pa != nullptr || pb != nullptr;
             pa = pa == nullptr ? nullptr : pa->next,
             pb = pb == nullptr ? nullptr : pb->next,
             prev = prev->next) {
            const int ai = pa == nullptr ? 0 : pa->val;
            const int bi = pb == nullptr ? 0 : pb->val;
            const int value = (ai + bi + carry) % 10;
            carry = (ai + bi + carry) / 10;
            prev->next = new ListNode(value); // 尾插法
        }
        if (carry > 0)
            prev->next = new ListNode(carry);
        return head.next;
    }
};
\end{Code}


\subsubsection{相关题目}

\begindot
\item Add Binary,见 \S \ref{sec:add-binary}
\myenddot


\subsection{Reverse Linked List II}
\label{sec:reverse-linked-list-ii}


\subsubsection{描述}
Reverse a linked list from position $m$ to $n$. Do it in-place and in one-pass.

For example:
Given \code{1->2->3->4->5->nullptr}, $m$ = 2 and $n$ = 4,

return \code{1->4->3->2->5->nullptr}.

Note:
Given m, n satisfy the following condition:
$1 \leq m \leq  n \leq $ length of list.


\subsubsection{分析}
这题非常繁琐,有很多边界检查,15分钟内做到bug free很有难度!


\subsubsection{代码}
\begin{Code}
// LeetCode, Reverse Linked List II
class Solution {
public:
    ListNode *reverseBetween(ListNode *head, int m, int n) {
        if(m >= n) return head;
        ListNode dummy(0);
        ListNode *h = &dummy;
        h->next = head;

        int count = 0;
        ListNode *p = h, *pm, *pn;
        for (; p; p = p->next) {
            if (count == m-1) pm = p;
            if (count == n) {
                pn = p;
                break;
            }
            count++;
        }
        p = pm;
        pm = pm->next;
        if (m == 1) head = pn;  // 若m=1,则pn就变为首节点

        p->next = pn;
        p = pm->next;
        pm->next = pn->next;

        ListNode *q = p->next; // pm->p->q
        while(pm != pn) {
            p->next = pm;
            pm = p;
            p = q;
            if(q) q = q->next;
        }

        return head;
    }
};
\end{Code}


\subsubsection{相关题目}

\begindot
\item 无
\myenddot


\subsection{Partition List}
\label{sec:partition-list}


\subsubsection{描述}
Given a linked list and a value $x$, partition it such that all nodes less than $x$ come before nodes greater than or equal to $x$.

You should preserve the original relative order of the nodes in each of the two partitions.

For example,
Given \code{1->4->3->2->5->2} and \code{x = 3}, return \code{1->2->2->4->3->5}.


\subsubsection{分析}
无


\subsubsection{代码}
\begin{Code}
// LeetCode, Partition List
class Solution {
public:
    ListNode* partition(ListNode* head, int x) {
        if (head == nullptr) return head;

        ListNode left_dummy(0); // 头结点
        ListNode right_dummy(0); // 头结点

        auto left_cur = &left_dummy;
        auto right_cur = &right_dummy;

        for (; head; head = head->next) {
            if (head->val < x) {
                left_cur->next = head;
                left_cur = head;
            } else {
                right_cur->next = head;
                right_cur = head;
            }
        }

        left_cur->next = right_dummy.next;
        right_cur->next = nullptr;

        return left_dummy.next;
    }
};
\end{Code}


\subsubsection{相关题目}

\begindot
\item 无
\myenddot


\subsection{Remove Duplicates from Sorted List}
\label{sec:remove-duplicates-from-sorted-list}


\subsubsection{描述}
Given a sorted linked list, delete all duplicates such that each element appear only once.

For example,

Given \code{1->1->2}, return \code{1->2}.

Given \code{1->1->2->3->3}, return \code{1->2->3}.


\subsubsection{分析}
无


\subsubsection{代码}
\begin{Code}
// LeetCode, Remove Duplicates from Sorted List
class Solution {
public:
    ListNode *deleteDuplicates(ListNode *head) {
        if (head == nullptr) return nullptr;
        ListNode * prev = head;
        ListNode *cur = head->next;
        while (cur != nullptr) {
            if (prev->val == cur->val) {
                ListNode* tmp = cur;
                cur = cur->next;
                prev->next = cur;
                delete tmp;
                continue;
            } else {
                prev = prev->next;
                cur = cur->next;
            }
        }
        return head;
    }
};
\end{Code}


\subsubsection{相关题目}

\begindot
\item Remove Duplicates from Sorted List II,见 \S \ref{sec:remove-duplicates-from-sorted-list-ii}
\myenddot


\subsection{Remove Duplicates from Sorted List II}
\label{sec:remove-duplicates-from-sorted-list-ii}


\subsubsection{描述}
Given a sorted linked list, delete all nodes that have duplicate numbers, leaving only distinct numbers from the original list.

For example,

Given \code{1->2->3->3->4->4->5}, return \code{1->2->5}.

Given \code{1->1->1->2->3}, return \code{2->3}.


\subsubsection{分析}
无


\subsubsection{代码}
\begin{Code}
// LeetCode, Remove Duplicates from Sorted List II
class Solution {
public:
    ListNode *deleteDuplicates(ListNode *head) {
        if (head == nullptr) return head;

        ListNode dummy(INT_MIN); // 头结点
        dummy.next = head;
        ListNode *prev = &dummy, *cur = head;
        while (cur != nullptr) {
            bool duplicated = false;
            while (cur->next != nullptr && cur->val == cur->next->val) {
                duplicated = true;
                ListNode *temp = cur;
                cur = cur->next;
                delete temp;
            }
            if (duplicated) { // 删除重复的最后一个元素
                ListNode *temp = cur;
                cur = cur->next;
                delete temp;
                continue;
            }
            prev->next = cur;
            prev = prev->next;
            cur = cur->next;
        }
        prev->next = cur;
        return dummy.next;
    }
};
\end{Code}


\subsubsection{相关题目}

\begindot
\item Remove Duplicates from Sorted List,见 \S \ref{sec:remove-duplicates-from-sorted-list}
\myenddot


\subsection{Rotate List}
\label{sec:rotate-list}


\subsubsection{描述}
Given a list, rotate the list to the right by $k$ places, where $k$ is non-negative.

For example:
Given \code{1->2->3->4->5->nullptr} and \code{k = 2}, return \code{4->5->1->2->3->nullptr}.


\subsubsection{分析}
先遍历一遍,得出链表长度$len$,注意$k$可能大于$len$,因此令$k \%= len$。将尾节点next指针指向首节点,形成一个环,接着往后跑$len-k$步,从这里断开,就是要求的结果了。


\subsubsection{代码}
\begin{Code}
// LeetCode, Remove Rotate List
class Solution {
public:
    ListNode *rotateRight(ListNode *head, int k) {
        if (head == nullptr || k == 0) return head;

        int len = 1;
        ListNode* p = head;
        while (p->next) { // 求长度
            len++;
            p = p->next;
        }
        k = len - k % len;

        p->next = head; // 首尾相连
        for(int step = 0; step < k; step++) {
            p = p->next;  //接着往后跑
        }
        head = p->next; // 新的首节点
        p->next = nullptr; // 断开环
        return head;
    }
};
\end{Code}


\subsubsection{相关题目}

\begindot
\item 无
\myenddot


\subsection{Remove Nth Node From End of List}
\label{sec:remove-nth-node-from-end-of-list}


\subsubsection{描述}
Given a linked list, remove the $n^{th}$ node from the end of list and return its head.

For example, Given linked list: \code{1->2->3->4->5}, and $n$ = 2.

After removing the second node from the end, the linked list becomes \code{1->2->3->5}.

Note:
\begindot
\item Given $n$ will always be valid.
\item Try to do this in one pass.
\myenddot


\subsubsection{分析}
设两个指针$p,q$,让$q$先走$n$步,然后$p$和$q$一起走,直到$q$走到尾节点,删除\fn{p->next}即可。


\subsubsection{代码}
\begin{Code}
// LeetCode, Remove Nth Node From End of List
class Solution {
public:
    ListNode *removeNthFromEnd(ListNode *head, int n) {
        ListNode dummy(0);
        dummy.next = head;
        ListNode *p = &dummy, *q = &dummy;

        for (int i = 0; i < n; i++) { // q先走n步
            q = q->next;
        }

        while(q->next) { // 一起走
            p = p->next;
            q = q->next;
        }
        ListNode *tmp = p->next;
        p->next = p->next->next;
        delete tmp;
        return dummy.next;
    }
};
\end{Code}


\subsubsection{相关题目}

\begindot
\item 无
\myenddot
