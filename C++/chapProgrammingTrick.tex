\chapter{编程技巧}
把较大的数组放在main函数外,作为全局变量,这样可以防止栈溢出,因为栈的大小是有限制的。

如果能够预估栈,队列的上限,则不要用\fn{std::stack, std::queue},使用数组来模拟,这样速度最快。

输入数据一般放在全局变量,且在运行过程中不要修改这些变量。

在判断两个浮点数a和b是否相等时,不要用\fn{a==b},应该判断二者之差的绝对值\fn{fabs(a-b)}是否小于某个阀值,例如\fn{1e-9}。

