\chapter{暴力枚举法}


\section{Subsets} %%%%%%%%%%%%%%%%%%%%%%%%%%%%%%
\label{sec:subsets}


\subsubsection{描述}
Given a set of distinct integers, $S$, return all possible subsets.

Note:
\begindot
\item Elements in a subset must be in non-descending order.
\item The solution set must not contain duplicate subsets.
\myenddot

For example, If \code{S = [1,2,3]}, a solution is:
\begin{Code}
[
  [3],
  [1],
  [2],
  [1,2,3],
  [1,3],
  [2,3],
  [1,2],
  []
]
\end{Code}


\subsection{增量构造法}
每个元素,都有两种选择,选或者不选。

\subsubsection{代码}
\begin{Code}
// LeetCode, Subsets
// 增量构造法,朴素深搜
class Solution {
public:
    vector<vector<int> > subsets(vector<int> &S) {
        vector<vector<int> > result;
        vector<int> cur;
        std::sort(S.begin(), S.end()); // 本题对顺序有要求,需要排序

        subsets(S, cur, 0, result);
        return result;
    }

private:
    static void subsets(const vector<int> &S, vector<int> &cur, int step,
            vector<vector<int> > &result) {
        if (step == S.size()) {
            result.push_back(cur);
            return;
        }
        // 不选S[step]
        subsets(S, cur, step + 1, result);
        // 选S[step]
        cur.push_back(S[step]);
        subsets(S, cur, step + 1, result);
        cur.pop_back();
    }
};
\end{Code}


\subsection{位向量法}
开一个位向量\fn{bool selected[n]},每个元素可以选或者不选。


\subsubsection{代码}
\begin{Code}
// LeetCode, Subsets
// 位向量法,也属于朴素深搜
class Solution {
public:
    vector<vector<int> > subsets(vector<int> &S) {
        vector<vector<int> > result;
        bool *selected = new bool[S.size()];
        std::sort(S.begin(), S.end()); // 本题对顺序有要求,需要排序

        subsets(S, selected, 0, result);
        delete[] selected;
        return result;
    }

private:
    static void subsets(const vector<int> &S, bool selected[], int step, 
            vector<vector<int> > &result) {
        if (step == S.size()) {
            vector<int> subset;
            for (int i = 0; i < S.size(); i++) {
                if (selected[i]) subset.push_back(S[i]);
            }
            result.push_back(subset);
            return;
        }
        // 不选S[step]
        selected[step] = false;
        subsets(S, selected, step + 1, result);
        // 选S[step]
        selected[step] = true;
        subsets(S, selected, step + 1, result);
    }
};
\end{Code}


\subsection{二进制法}
本方法的前提是:集合的元素不超过int位数。用一个int整数表示位向量,第$i$位为1,则表示选择$S[i]$,为0则不选择。例如\fn{S=\{A,B,C,D\}},则\fn{0110=6}表示子集\fn{\{B,C\}}。

这种方法最巧妙。因为它不仅能生成子集,还能方便的表示集合的并、交、差等集合运算。设两个集合的位向量分别为$B_1$和$B_2$,则$B_1|B_2, B_1 \& B_2, B_1 \^ B_2$分别对应集合的并、交、对称差。

\begin{Code}
// LeetCode, Subsets
// 二进制法
class Solution {
public:
    vector<vector<int> > subsets(vector<int> &S) {
        vector<vector<int> > result;
        std::sort(S.begin(), S.end()); // 本题对顺序有要求,需要排序
        const size_t n = S.size();
        vector<int> v;

        for (size_t i = 0; i < 1 << n; i++) {
            for (size_t j = 0; j < n; j++) {
                if (i & 1 << j) v.push_back(S[j]);
            }
            result.push_back(v);
            v.clear();
        }
        return result;
    }
};
\end{Code}


\subsubsection{代码}
\begin{Code}

\end{Code}

\subsubsection{相关题目}
\begindot
\item Subsets II,见 \S \ref{sec:subsets-ii}
\myenddot


\section{Subsets II} %%%%%%%%%%%%%%%%%%%%%%%%%%%%%%
\label{sec:subsets-ii}


\subsubsection{描述}
Given a collection of integers that might contain duplicates, $S$, return all possible subsets.

Note:

Elements in a subset must be in non-descending order.
The solution set must not contain duplicate subsets.
For example,
If \fn{S = [1,2,2]}, a solution is:
\begin{Code}
[
  [2],
  [1],
  [1,2,2],
  [2,2],
  [1,2],
  []
]
\end{Code}


\subsubsection{分析}
这题有重复元素,但本质上,跟上一题很类似,上一题中元素没有重复,相当于每个元素只能选0或1次,这里扩充到了每个元素可以选0到若干次而已。


\subsubsection{代码}
\begin{Code}
// LeetCode, Subsets II
class Solution {
public:
    vector<vector<int> > subsetsWithDup(vector<int> &S) {
        vector<vector<int> > result;
        std::sort(S.begin(), S.end()); // 本题对顺序有要求,需要排序
        std::vector<int> count(S.back() - S.front() + 1, 0);
        // 计算所有元素的个数
        for(auto i = S.begin(); i != S.end(); i++) {
            count[*i - S[0]]++;
        }

        // 每个元素选择了多少个
        std::vector<int> selected(S.back() - S.front() + 1, -1);

        subsets(S, count, selected, 0, result);
        return result;
    }

private:
    static void subsets(const vector<int> &S, vector<int> &count,
            vector<int> &selected, size_t step, vector<vector<int> > &result) {
        if (step == count.size()) {
            vector<int> subset;
            for(size_t i = 0; i < selected.size(); i++) {
                for (int j = 0; j < selected[i]; j++) {
                    subset.push_back(i+S[0]);
                }
            }
            result.push_back(subset);
            return;
        }

        for (int i = 0; i <= count[step]; i++) {
            selected[step] = i;
            subsets(S, count, selected, step + 1, result);
        }
    }
};
\end{Code}


\subsubsection{相关题目}
\begindot
\item Subsets,见 \S \ref{sec:subsets}
\myenddot
