\chapter{树}

LeetCode 上二叉树的节点定义如下:
\begin{Code}
// 树的节点
struct TreeNode {
    int val;
    TreeNode *left;
    TreeNode *right;
    TreeNode(int x) : val(x), left(nullptr), right(nullptr) { }
};
\end{Code}


\section{二叉树的遍历} %%%%%%%%%%%%%%%%%%%%%%%%%%%%%%

树的遍历有两类:深度优先遍历和宽度优先遍历。深度优先遍历又可分为两种:先根(次序)遍历和后根(次序)遍历。

树的先根遍历是:先访问树的根结点,然后依次先根遍历根的各棵子树。树的先跟遍历的结果与对应二叉树(孩子兄弟表示法)的先序遍历的结果相同。

树的后根遍历是:先依次后根遍历树根的各棵子树,然后访问根结点。树的后跟遍历的结果与对应二叉树的中序遍历的结果相同。

二叉树的先根遍历有:\textbf{先序遍历}(root->left->right),root->right->left;后根遍历有:\textbf{后序遍历}(left->right->root),right->left->root;二叉树还有个一般的树没有的遍历次序,\textbf{中序遍历}(left->root->right)。


\subsection{Binary Tree Level Order Traversal}
\label{sec:binary-tree-tevel-order-traversal}


\subsubsection{描述}
Given a binary tree, return the level order traversal of its nodes' values. (ie, from left to right, level by level).

For example:
Given binary tree \code{\{3,9,20,\#,\#,15,7\}},
\begin{Code}
    3
   / \
  9  20
    /  \
   15   7
\end{Code}
return its level order traversal as:
\begin{Code}
[
  [3],
  [9,20],
  [15,7]
]
\end{Code}


\subsubsection{分析}
无


\subsubsection{代码}
\begin{Code}
//LeetCode, Binary Tree Level Order Traversal
class Solution {
public:
    vector<vector<int> > levelOrder(TreeNode *root) {
        vector<vector<int> > result;
        if(root == nullptr) return result;

        queue<TreeNode*> current_level, next_level;
        vector<int> level; // elments in level level

        current_level.push(root);
        while (!current_level.empty()) {
            while (!current_level.empty()) {
                TreeNode* node = current_level.front();
                current_level.pop();
                level.push_back(node->val);
                if (node->left != nullptr) next_level.push(node->left);
                if (node->right != nullptr) next_level.push(node->right);
            }
            result.push_back(level);
            level.clear();
            swap(next_level, current_level); //!!! how to use swap
        }
        return result;
    }
};
\end{Code}


\subsubsection{相关题目}
\begindot
\item Binary Tree Level Order Traversal II,见 \S \ref{sec:binary-tree-tevel-order-traversal-ii}
\item Binary Tree Zigzag Level Order Traversal,见 \S \ref{sec:binary-tree-zigzag-level-order-traversal}
\myenddot


\subsection{Binary Tree Level Order Traversal II}
\label{sec:binary-tree-tevel-order-traversal-ii}


\subsubsection{描述}
Given a binary tree, return the bottom-up level order traversal of its nodes' values. (ie, from left to right, level by level from leaf to root).

For example:
Given binary tree \code{\{3,9,20,\#,\#,15,7\}},
\begin{Code}
    3
   / \
  9  20
    /  \
   15   7
\end{Code}
return its bottom-up level order traversal as:
\begin{Code}
[
  [15,7]
  [9,20],
  [3],
]
\end{Code}


\subsubsection{分析}
在Binary Tree Level Order Traversal I(见\S \ref{sec:binary-tree-tevel-order-traversal})的基础上,用一个list作为栈,每次在头部插入,就可以实现倒着输出


\subsubsection{代码}
\begin{Code}
//LeetCode, Binary Tree Level Order Traversal II
//在Binary Tree Level Order Traversal I的基础上,用一个list作为栈
//每次在头部插入,就可以实现倒着输出
class Solution {
public:
    vector<vector<int> > levelOrderBottom(TreeNode *root) {
        list<vector<int> > tmp_result;

        queue<TreeNode *> q;
        q.push(root);
        q.push(nullptr); // level separator

        vector<int> level;  // elements in one level
        while(!q.empty()) {
            TreeNode *cur = q.front(); q.pop();

            if(cur) {
                level.push_back(cur->val);
                if(cur->left) q.push(cur->left);
                if(cur->right) q.push(cur->right);
            } else {
                if(level.size() > 0) {
                    tmp_result.push_front(level);
                    level.clear();
                    q.push(nullptr);
                }
            }
        }

        vector<vector<int> > ret;
        for(auto iter = tmp_result.begin(); iter != tmp_result.end(); ++iter) {
            ret.push_back(*iter);
        }
        return ret;
    }
};
\end{Code}


\subsubsection{相关题目}
\begindot
\item Binary Tree Level Order Traversal,见 \S \ref{sec:binary-tree-tevel-order-traversal}
\item Binary Tree Zigzag Level Order Traversal,见 \S \ref{sec:binary-tree-zigzag-level-order-traversal}
\myenddot


\subsection{Binary Tree Zigzag Level Order Traversal}
\label{sec:binary-tree-zigzag-level-order-traversal}


\subsubsection{描述}
Given a binary tree, return the zigzag level order traversal of its nodes' values. (ie, from left to right, then right to left for the next level and alternate between).

For example:
Given binary tree \code{{3,9,20,\#,\#,15,7}},
\begin{Code}
    3
   / \
  9  20
    /  \
   15   7
\end{Code}
return its zigzag level order traversal as:
\begin{Code}
[
  [3],
  [20,9],
  [15,7]
]
\end{Code}


\subsubsection{分析}
广度优先遍历,用一个bool记录是从左到右还是从右到左,每一层结束就翻转一下。


\subsubsection{代码}
\begin{Code}
//LeetCode, Binary Tree Zigzag Level Order Traversal
//广度优先遍历,用一个bool记录是从左到右还是从右到左,每一层结束就翻转一下。
class Solution {
public:
    vector<vector<int> > zigzagLevelOrder(TreeNode *root) {
        vector<vector<int> > result;
        if (nullptr == root) return result;

        queue<TreeNode*> q;
        bool l2r = true;  //left to right
        vector<int> level;  // one level's elements

        q.push(root);
        q.push(nullptr);  // level separator
        while (!q.empty()) {
            TreeNode *cur = q.front();
            q.pop();
            if (cur) {
                level.push_back(cur->val);
                if (cur->left) q.push(cur->left);
                if (cur->right) q.push(cur->right);
            } else {
                if (l2r) {
                    result.push_back(level);
                } else {
                    vector<int> tmp;
                    for (auto iter = level.rbegin(); iter != level.rend(); ++iter) {
                        tmp.push_back(*iter);
                    }
                    result.push_back(tmp);
                }
                level.clear();
                l2r = !l2r;

                if (q.size() > 0) q.push(nullptr);
            }
        }

        return result;
    }
};
\end{Code}


\subsubsection{相关题目}
\begindot
\item Binary Tree Level Order Traversal,见 \S \ref{sec:binary-tree-tevel-order-traversal}
\item Binary Tree Level Order Traversal II,见 \S \ref{sec:binary-tree-tevel-order-traversal-ii}
\myenddot


\subsection{Binary Tree Inorder Traversal}
\label{sec:binary-tree-inorder-traversal}


\subsubsection{描述}
Given a binary tree, return the inorder traversal of its nodes' values.

For example:
Given binary tree \code{\{1,\#,2,3\}},
\begin{Code}
 1
  \
   2
  /
 3
\end{Code}
return \code{[1,3,2]}.

Note: Recursive solution is trivial, could you do it iteratively?


\subsubsection{分析}
不用递归,可用栈,Morris中序遍历或者线索二叉树哦。


\subsubsection{代码}

栈
\begin{Code}
// LeetCode, Binary Tree Inorder Traversal
// 使用栈
class Solution {
public:
    vector<int> inorderTraversal(TreeNode *root) {
        vector<int> result;
        const TreeNode *p = root;
        stack<const TreeNode *> s;

        while (!s.empty() || p != nullptr) {
            if (p != nullptr) {
                s.push(p);
                p = p->left;
            } else {
                p = s.top();
                s.pop();
                result.push_back(p->val);
                p = p->right;
            }
        }
        return result;
    }
};
\end{Code}

Morris中序遍历
\begin{Code}
// LeetCode, Binary Tree Inorder Traversal
// Morris中序遍历
class Solution {
public:
    vector<int> inorderTraversal(TreeNode *root) {
        vector<int> result;
        TreeNode* prev = nullptr;
        TreeNode* cur = root;

        while (cur != nullptr) {
            if (cur->left == nullptr) {
                result.push_back(cur->val);
                prev = cur;
                cur = cur->right;
            } else {
                auto node = cur->left;

                while (node->right != nullptr && node->right != cur)
                    node = node->right;

                if (node->right == nullptr) {
                    node->right = cur;
                    //prev = cur; 不能有这句,因为cur还没有被访问
                    cur = cur->left;
                } else {
                    result.push_back(cur->val);
                    node->right = nullptr;
                    prev = cur;
                    cur = cur->right;
                }
            }
        }
        return result;
    }
};
\end{Code}


\subsubsection{相关题目}
\begindot
\item Recover Binary Search Tree,见 \S \ref{sec:recover-binary-search-tree}
\myenddot


\subsection{Recover Binary Search Tree}
\label{sec:recover-binary-search-tree}


\subsubsection{描述}
Two elements of a binary search tree (BST) are swapped by mistake.

Recover the tree without changing its structure.

Note: A solution using $O(n)$ space is pretty straight forward. Could you devise a constant space solution?


\subsubsection{分析}
$O(n)$空间的解法是,开一个指针数组,中序遍历,将节点指针依次存放到数组里,然后寻找两处逆向的位置,先从前往后找第一个逆序的位置,然后从后往前找第二个逆序的位置,交换这两个指针的值。

中序遍历一般需要用到栈,空间也是$O(n)$的,如何才能不使用栈?Morris中序遍历。


\subsubsection{代码}

\begin{Code}
// LeetCode, Recover Binary Search Tree
// Morris中序遍历
class Solution {
public:
    void recoverTree(TreeNode* root) {
        pair<TreeNode*, TreeNode*> broken;
        TreeNode* prev = nullptr;
        TreeNode* cur = root;

        while (cur != nullptr) {
            if (cur->left == nullptr) {
                detect(broken, prev, cur);
                prev = cur;
                cur = cur->right;
            } else {
                auto node = cur->left;

                while (node->right != nullptr && node->right != cur)
                    node = node->right;

                if (node->right == nullptr) {
                    node->right = cur;
                    //prev = cur; 不能有这句!因为cur还没有被访问
                    cur = cur->left;
                } else {
                    detect(broken, prev, cur);
                    node->right = nullptr;
                    prev = cur;
                    cur = cur->right;
                }
            }
        }

        swap(broken.first->val, broken.second->val);
    }

    void detect(pair<TreeNode*, TreeNode*>& broken, TreeNode* prev,
            TreeNode* current) {
        if (prev != nullptr && prev->val > current->val) {
            if (broken.first == nullptr) {
                broken.first = prev;
            } //不能用else,例如 {0,1},会导致最后 swap时second为nullptr,
              //会 Runtime Error
            broken.second = current;
        }
    }
};
\end{Code}


\subsubsection{相关题目}
\begindot
\item Binary Tree Inorder Traversal,见 \S \ref{sec:binary-tree-inorder-traversal}
\myenddot


\subsection{Same Tree}
\label{sec:same-tree}


\subsubsection{描述}
Given two binary trees, write a function to check if they are equal or not.

Two binary trees are considered equal if they are structurally identical and the nodes have the same value.


\subsubsection{分析}
无


\subsubsection{代码}
递归版
\begin{Code}
// LeetCode, Same Tree
// 迭代
class Solution {
public:
    bool isSameTree(TreeNode *p, TreeNode *q) {
        if (!p && !q) return true;   // 终止条件
        if (!p || !q) return false;  // 剪纸
        return p->val == q->val      // 三方合并
                && isSameTree(p->left, q->left)
                && isSameTree(p->right, q->right);
    }
};
\end{Code}

迭代版
\begin{Code}
// LeetCode, Same Tree
// 迭代
class Solution {
public:
    bool isSameTree(TreeNode *p, TreeNode *q) {
        stack<TreeNode*> s;
        s.push(p);
        s.push(q);

        while(!s.empty()) {
            p = s.top(); s.pop();
            q = s.top(); s.pop();

            if (!p && !q) continue;
            if (!p || !q) return false;
            if (p->val != q->val) return false;

            s.push(p->left);
            s.push(q->left);

            s.push(p->right);
            s.push(q->right);
        }
        return true;
    }
};
\end{Code}


\subsubsection{相关题目}
\begindot
\item Symmetric Tree,见 \S \ref{sec:symmetric-tree}
\myenddot


\subsection{Symmetric Tree}
\label{sec:symmetric-tree}


\subsubsection{描述}
Given two binary trees, write a function to check if they are equal or not.

Two binary trees are considered equal if they are structurally identical and the nodes have the same value.


\subsubsection{分析}
无


\subsubsection{代码}
递归版
\begin{Code}
// LeetCode, Symmetric Tree
// 递归版
class Solution {
public:
    bool isSymmetric(TreeNode *root) {
        return root ? isSymmetric(root->left, root->right) : true;
    }
    bool isSymmetric(TreeNode *left, TreeNode *right) {
        if (!left && !right) return true;   // 终止条件
        if (!left || !right) return false;  // 终止条件
        return left->val != right->val      // 三方合并
                && isSymmetric(left->left, right->right)
                && isSymmetric(left->right, right->left);
    }
};
\end{Code}

迭代版
\begin{Code}
// LeetCode, Symmetric Tree
// 迭代版
class Solution {
public:
    bool isSymmetric (TreeNode* root) {
        if (!root) return true;

        stack<TreeNode*> s;
        s.push(root->left);
        s.push(root->right);

        while (!s.empty ()) {
            auto p = s.top (); s.pop();
            auto q = s.top (); s.pop();

            if (!p && !q) continue;
            if (!p || !q) return false;
            if (p->val != q->val) return false;

            s.push(p->left);
            s.push(q->right);

            s.push(p->right);
            s.push(q->left);
        }

        return true;
    }
};
\end{Code}


\subsubsection{相关题目}
\begindot
\item Same Tree,见 \S \ref{sec:same-tree}
\myenddot


\subsection{Balanced Binary Tree}
\label{sec:balanced-binary-tree}


\subsubsection{描述}
Given a binary tree, determine if it is height-balanced.

For this problem, a height-balanced binary tree is defined as a binary tree in which the depth of the two subtrees of every node never differ by more than 1.


\subsubsection{分析}
无


\subsubsection{代码}
\begin{Code}
// LeetCode, Balanced Binary Tree
class Solution {
public:
    bool isBalanced (TreeNode* root) {
        return balancedHeight (root) >= 0;
    }

    /**
     * Returns the height of `root` if `root` is a balanced tree,
     * otherwise, returns `-1`.
     */
    int balancedHeight (TreeNode* root) {
        if (root == nullptr) return 0;  // 终止条件

        int lhs = balancedHeight (root->left);
        int rhs = balancedHeight (root->right);

        if (lhs < 0 || rhs < 0 || abs(lhs - rhs) > 1) return -1;  // 剪枝

        return max(lhs, rhs) + 1; // 三方合并
    }
};
\end{Code}


\subsubsection{相关题目}
\begindot
\item 无
\myenddot


\subsection{Flatten Binary Tree to Linked List}
\label{sec:flatten-binary-tree-to-linked-list}


\subsubsection{描述}
Given a binary tree, flatten it to a linked list in-place.

For example, Given
\begin{Code}
         1
        / \
       2   5
      / \   \
     3   4   6
\end{Code}

The flattened tree should look like:
\begin{Code}
   1
    \
     2
      \
       3
        \
         4
          \
           5
            \
             6
\end{Code}

\subsubsection{分析}
无


\subsubsection{代码}
递归版
\begin{Code}
// LeetCode, Flatten Binary Tree to Linked List,递归版
class Solution {
public:
    void flatten(TreeNode *root) {
        if (root == nullptr) return;  // 终止条件
        //1.flat the left subtree
        if (root->left) flatten(root->left);

        //2.flatten the right subtree
        if (root->right) flatten(root->right);

        //3.if no left return
        if (nullptr == root->left) return;

        //4.insert left sub tree between root and root->right
        //4.1.find the last node in left
        TreeNode *p = root->left;
        while(p->right) p = p->right;
        p->right = root->right;
        //4.2.connect right sub tree after left sub tree
        p->right = root->right;
        //4.3.move left sub tree to the root's right sub tree
        root->right = root->left;
        root->left = nullptr;
    }
};
\end{Code}

迭代版
\begin{Code}
// LeetCode, Flatten Binary Tree to Linked List
// 迭代版
class Solution {
public:
    void flatten(TreeNode* root) {
        if (root == nullptr) return;

        stack<TreeNode*> s;
        s.push(root);

        while (!s.empty()) {
            auto p = s.top();
            s.pop();

            if (p->right)
                s.push(p->right);
            if (p->left)
                s.push(p->left);

            p->left = nullptr;
            if (!s.empty())
                p->right = s.top();
        }
    }
};
\end{Code}


\subsubsection{相关题目}
\begindot
\item 无
\myenddot


\section{二叉查找树} %%%%%%%%%%%%%%%%%%%%%%%%%%%%%%


\subsection{Unique Binary Search Trees}
\label{sec:unique-binary-search-trees}


\subsubsection{描述}
Given $n$, how many structurally unique BST's (binary search trees) that store values $1...n$?

For example,
Given $n = 3$, there are a total of 5 unique BST's.
\begin{Code}
   1         3     3      2      1
    \       /     /      / \      \
     3     2     1      1   3      2
    /     /       \                 \
   2     1         2                 3
\end{Code}

\subsubsection{分析}
如果把上例的顺序改一下,就可以看出规律了。
\begin{Code}
 1       1           2          3       3
  \       \         / \        /       / 
   3       2       1   3      2       1
  /         \                /         \
2            3              1           2
\end{Code}

比如,以1为根的树的个数,等于左子树的个数乘以右子树的个数,左子树是0个元素的树,右子树是2个元素的树。以2为根的树的个数,等于左子树的个数乘以右子树的个数,左子树是1个元素的树,右子树也是1个元素的树。依此类推。

当数组为 $1,2,3,...,n$时,基于以下原则的构建的BST树具有唯一性:
\textbf{以i为根节点的树,其左子树由[1, i-1]构成, 其右子树由[i+1, n]构成。}

定义$f(i)$为以$[1,i]$能产生的Unique Binary Search Tree的数目,则

如果数组为空,毫无疑问,只有一种BST,即空树,$f(0)=1$。

如果数组仅有一个元素{1},只有一种BST,单个节点,$f(1)=1$。

如果数组有两个元素{1,2}, 那么有如下两种可能
\begin{Code}
1             2
  \          /
    2      1
\end{Code}

\begin{eqnarray}
f(2) &=& f(0) * f(1)   \text{ ,1为根的情况} \nonumber \\
     &+& f(1) * f(0)   \text{ ,2为根的情况} \nonumber
\end{eqnarray}

再看一看3个元素的数组,可以发现BST的取值方式如下:
\begin{eqnarray}
f(3) &=& f(0) * f(2)   \text{ ,1为根的情况} \nonumber \\
     &+& f(1) * f(1)   \text{ ,2为根的情况} \nonumber \\
     &+& f(2) * f(0)   \text{ ,3为根的情况} \nonumber
\end{eqnarray}

所以,由此观察,可以得出$f$的递推公式为
$$
f(i) = \sum_{k=1}^{i} f(k-1) \times f(i-k)
$$
至此,问题划归为一维动态规划。


\subsubsection{代码}

\begin{Code}
// LeetCode, Unique Binary Search Trees
class Solution {
public:
    int numTrees(int n) {
        vector<int> f(n + 1, 0);

        f[0] = 1;
        f[1] = 1;
        for (int i = 2; i <= n; ++i) {
            for (int k = 1; k <= i; ++k)
                f[i] += f[k-1] * f[i - k];
        }

        return f[n];
    }
};
\end{Code}


\subsubsection{相关题目}
\begindot
\item Unique Binary Search Trees II,见 \S \ref{sec:unique-binary-search-trees-ii}
\myenddot


\subsection{Unique Binary Search Trees II}
\label{sec:unique-binary-search-trees-ii}


\subsubsection{描述}
Given $n$, generate all structurally unique BST's (binary search trees) that store values 1...n.

For example,
Given $n = 3$, your program should return all 5 unique BST's shown below.
\begin{Code}
   1         3     3      2      1
    \       /     /      / \      \
     3     2     1      1   3      2
    /     /       \                 \
   2     1         2                 3
\end{Code}


\subsubsection{分析}
见前面一题。


\subsubsection{代码}

\begin{Code}
// LeetCode, Unique Binary Search Trees II
class Solution {
public:
    vector<TreeNode *> generateTrees(int n) {
        if (n == 0) return generate(1, 0);
        return generate(1, n);
    }
private:
    vector<TreeNode *> generate(int start, int end) {
        vector<TreeNode*> subTree;
        if (start > end) {
            subTree.push_back(nullptr);
            return subTree;
        }
        for (int k = start; k <= end; k++) {
            vector<TreeNode*> leftSubs = generate(start, k - 1);
            vector<TreeNode*> rightSubs = generate(k + 1, end);
            for (auto i = leftSubs.begin(); i < leftSubs.end(); i++) {
                for (auto j = rightSubs.begin(); j < rightSubs.end(); j++) {
                    TreeNode *node = new TreeNode(k);
                    node->left = *i;
                    node->right = *j;
                    subTree.push_back(node);
                }
            }
        }
        return subTree;
    }
};
\end{Code}


\subsubsection{相关题目}
\begindot
\item Unique Binary Search Trees,见 \S \ref{sec:unique-binary-search-trees}
\myenddot


\subsection{Validate Binary Search Tree}
\label{sec:validate-binary-search-tree}


\subsubsection{描述}
Given a binary tree, determine if it is a valid binary search tree (BST).

Assume a BST is defined as follows:
\begindot
\item The left subtree of a node contains only nodes with keys less than the node's key.
\item The right subtree of a node contains only nodes with keys greater than the node's key.
\item Both the left and right subtrees must also be binary search trees.
\myenddot


\subsubsection{分析}


\subsubsection{代码}

\begin{Code}
// LeetCode, Validate Binary Search Tree
class Solution {
public:
    bool isValidBST(TreeNode* root) {
        return isValidBST(root, INT_MIN, INT_MAX);
    }

    bool isValidBST(TreeNode* root, int lower, int upper) {
        if (root == nullptr) return true;

        return root->val > lower && root->val < upper
                && isValidBST(root->left, lower, root->val)
                && isValidBST(root->right, root->val, upper);
    }
};
\end{Code}


\subsubsection{相关题目}
\begindot
\item Validate Binary Search Tree,见 \S \ref{sec:validate-binary-search-tree}
\myenddot


\subsection{Convert Sorted Array to Binary Search Tree}
\label{sec:convert-sorted-array-to-binary-search-tree}


\subsubsection{描述}
Given an array where elements are sorted in ascending order, convert it to a height balanced BST.


\subsubsection{分析}
二分法。


\subsubsection{代码}

\begin{Code}
// LeetCode, Convert Sorted Array to Binary Search Tree
// 分治法
class Solution {
public:
    TreeNode* sortedArrayToBST (vector<int>& num) {
        return sortedArrayToBST(num.begin(), num.end());
    }

    template<typename RandomAccessIterator>
    TreeNode* sortedArrayToBST (RandomAccessIterator first,
            RandomAccessIterator last) {
        const auto length = distance(first, last);

        if (length == 0) return nullptr;  // 终止条件
        if (length == 1) return new TreeNode(*first); // 收敛条件

        // 三方合并
        auto mid = first + length / 2;
        TreeNode* root = new TreeNode (*mid);
        root->left = sortedArrayToBST(first, mid);
        root->right = sortedArrayToBST(mid + 1, last);

        return root;
    }
};
\end{Code}


\subsubsection{相关题目}
\begindot
\item Convert Sorted List to Binary Search Tree,见 \S \ref{sec:convert-sorted-list-to-binary-search-tree}
\myenddot


\subsection{Convert Sorted List to Binary Search Tree}
\label{sec:convert-sorted-list-to-binary-search-tree}


\subsubsection{描述}
Given a singly linked list where elements are sorted in ascending order, convert it to a height balanced BST.


\subsubsection{分析}
这题与上一题类似,但是单链表不能随机访问,而自顶向下的二分法必须需要RandomAccessIterator,因此前面的方法不适用本题。

存在一种自底向上(bottom-up)的方法,见\myurl{http://leetcode.com/2010/11/convert-sorted-list-to-balanced-binary.html}

\subsubsection{代码}

分治法,类似于 Convert Sorted Array to Binary Search Tree,自顶向下,复杂度$O(n\log n)$。
\begin{Code}
// LeetCode, Convert Sorted List to Binary Search Tree
// 分治法,类似于 Convert Sorted Array to Binary Search Tree,
// 自顶向下,复杂度O(nlogn)
class Solution {
public:
    TreeNode* sortedListToBST (ListNode* head) {
        return sortedListToBST (head, listLength (head));
    }

    TreeNode* sortedListToBST (ListNode* head, int len) {
        if (len == 0) return nullptr;
        if (len == 1) return new TreeNode (head->val);

        TreeNode* root = new TreeNode (nth_node (head, len / 2 + 1)->val);
        root->left = sortedListToBST (head, len / 2);
        root->right = sortedListToBST (nth_node (head, len / 2 + 2), 
                (len - 1) / 2);

        return root;
    }

    int listLength (ListNode* node) {
        int n = 0;

        while(node) {
            ++n;
            node = node->next;
        }

        return n;
    }

    ListNode* nth_node (ListNode* node, int n) {
        while (--n)
            node = node->next;

        return node;
    }
};
\end{Code}

自底向上,复杂度$O(n)$。
\begin{Code}
// LeetCode, Convert Sorted List to Binary Search Tree
// bottom-up,复杂度O(nlogn)
class Solution {
public:
    TreeNode *sortedListToBST(ListNode *head) {
        int len = 0;
        ListNode *p = head;
        while (p) {
            len++;
            p = p->next;
        }
        return sortedListToBST(head, 0, len - 1);
    }
private:
    TreeNode* sortedListToBST(ListNode*& list, int start, int end) {
        if (start > end) return nullptr;

        int mid = start + (end - start) / 2;
        TreeNode *leftChild = sortedListToBST(list, start, mid - 1);
        TreeNode *parent = new TreeNode(list->val);
        parent->left = leftChild;
        list = list->next;
        parent->right = sortedListToBST(list, mid + 1, end);
        return parent;
    }
};
\end{Code}


\subsubsection{相关题目}
\begindot
\item Convert Sorted Array to Binary Search Tree,见 \S \ref{sec:convert-sorted-array-to-binary-search-tree}
\myenddot


\section{二叉树的深度} %%%%%%%%%%%%%%%%%%%%%%%%%%%%%%


\subsection{Minimum Depth of Binary Tree}
\label{sec:minimum-depth-of-binary-tree}


\subsubsection{描述}
Given a binary tree, find its minimum depth.

The minimum depth is the number of nodes along the shortest path from the root node down to the nearest leaf node.


\subsubsection{分析}
无

\subsubsection{代码}

递归版
\begin{Code}
// LeetCode, Minimum Depth of Binary Tree
// 递归版
class Solution {
public:
    int minDepth(TreeNode *root) {
        if (root == nullptr) return 0;
        else {
            d = INT_MAX;
            minDepth(root, 1);
            return d;
        }
    }
private:
    int d;
    void minDepth(TreeNode *root, int cur) {
        if (root != nullptr) {
            if (root->left == nullptr && root->right == nullptr) { // 叶子节点
                d = min(cur, d);
                return;
            }
            if (cur < d) { // 剪枝
                minDepth(root->left, cur+1);
                minDepth(root->right, cur+1);
            }
        }
    }
};
\end{Code}

迭代版
\begin{Code}
// LeetCode, Minimum Depth of Binary Tree
// 迭代版
class Solution {
public:
    int minDepth(TreeNode* root) {
        if (root == nullptr)
            return 0;

        int result = INT_MAX;

        stack<pair<TreeNode*, int>> s;
        s.push(make_pair(root, 1));

        while (!s.empty()) {
            auto node = s.top().first;
            auto depth = s.top().second;
            s.pop();

            if (node->left == nullptr && node->right == nullptr)
                result = min(result, depth);

            if (node->left && result > depth) // 深度控制,剪枝
                s.push(make_pair(node->left, depth + 1));

            if (node->right && result > depth) // 深度控制,剪枝
                s.push(make_pair(node->right, depth + 1));
        }

        return result;
    }
};
\end{Code}

\subsubsection{相关题目}
\begindot
\item Maximum Depth of Binary Tree,见 \S \ref{sec:maximum-depth-of-binary-tree}
\myenddot


\subsection{Maximum Depth of Binary Tree}
\label{sec:maximum-depth-of-binary-tree}


\subsubsection{描述}
Given a binary tree, find its maximum depth.

The maximum depth is the number of nodes along the longest path from the root node down to the farthest leaf node.


\subsubsection{分析}
无

\subsubsection{代码}

\begin{Code}
// LeetCode, Maximum Depth of Binary Tree
class Solution {
public:
    int maxDepth(TreeNode *root) {
        if (root == nullptr) return 0;

        int lmax = maxDepth(root->left);
        int rmax = maxDepth(root->right);
        return max(lmax, rmax) + 1;
    }
};
\end{Code}


\subsubsection{相关题目}
\begindot
\item Minimum Depth of Binary Tree,见 \S \ref{sec:minimum-depth-of-binary-tree}
\myenddot


\section{二叉树的构建} %%%%%%%%%%%%%%%%%%%%%%%%%%%%%%


\subsection{Construct Binary Tree from Preorder and Inorder Traversal}
\label{sec:construct-binary-tree-from-preorder-and-inorder-traversal}


\subsubsection{描述}
Given preorder and inorder traversal of a tree, construct the binary tree.

Note:
You may assume that duplicates do not exist in the tree.


\subsubsection{分析}
无


\subsubsection{代码}
\begin{Code}
// LeetCode, Construct Binary Tree from Preorder and Inorder Traversal
class Solution {
public:
    TreeNode* buildTree(vector<int>& preorder, vector<int>& inorder) {
        return buildTree(begin(preorder), end(preorder),
                begin(inorder), end(inorder));
    }

    template<typename InputIterator>
    TreeNode* buildTree(InputIterator pre_first, InputIterator pre_last,
            InputIterator in_first, InputIterator in_last) {
        if (pre_first == pre_last) return nullptr;
        if (in_first == in_last) return nullptr;

        auto root = new TreeNode(*pre_first);

        auto inRootPos = find(in_first, in_last, *pre_first);
        auto leftSize = distance(in_first, inRootPos);

        root->left = buildTree(next(pre_first), next(pre_first,
                leftSize + 1), in_first, next(in_first, leftSize));
        root->right = buildTree(next(pre_first, leftSize + 1), pre_last,
                next(inRootPos), in_last);

        return root;
    }
};
\end{Code}


\subsubsection{相关题目}
\begindot
\item Construct Binary Tree from Inorder and Postorder Traversal,见 \S \ref{sec:construct-binary-tree-from-inorder-and-postorder-traversal}
\myenddot


\subsection{Construct Binary Tree from Inorder and Postorder Traversal}
\label{sec:construct-binary-tree-from-inorder-and-postorder-traversal}


\subsubsection{描述}
Given inorder and postorder traversal of a tree, construct the binary tree.

Note:
You may assume that duplicates do not exist in the tree.


\subsubsection{分析}
无


\subsubsection{代码}
\begin{Code}
// LeetCode, Construct Binary Tree from Inorder and Postorder Traversal
class Solution {
public:
    TreeNode* buildTree(vector<int>& inorder, vector<int>& postorder) {
        return buildTree(begin(inorder), end(inorder),
                begin(postorder), end(postorder));
    }

    template<typename BidiIt>
    TreeNode* buildTree(BidiIt in_first, BidiIt in_last,
            BidiIt post_first, BidiIt post_last) {
        if (in_first ==in_last) return nullptr;
        if (post_first == post_last) return nullptr;

        const auto val = *prev(post_last);
        TreeNode* root = new TreeNode(val);

        auto in_root_pos = find(in_first, in_last, val);
        auto left_size = distance(in_first, in_root_pos);
        auto post_left_last = next(post_first, left_size);

        root->left = buildTree(in_first, in_root_pos, post_first, post_left_last);
        root->right = buildTree(next(in_root_pos), in_last, post_left_last,
                prev(post_last));

        return root;
    }
};
\end{Code}


\subsubsection{相关题目}
\begindot
\item Construct Binary Tree from Preorder and Inorder Traversal,见 \S \ref{sec:construct-binary-tree-from-preorder-and-inorder-traversal}
\myenddot


\section{二叉树的DFS} %%%%%%%%%%%%%%%%%%%%%%%%%%%%%%
二叉树的先序、中序、后序遍历都可以看做是DFS,此外还有其他遍历顺序,共有$3!=6$种。其他3种顺序是 \fn{root->r->l,r->root->l, r->l->root}。


\subsection{Path Sum}
\label{sec:path-sum}


\subsubsection{描述}
Given a binary tree and a sum, determine if the tree has a root-to-leaf path such that adding up all the values along the path equals the given sum.

For example:
Given the below binary tree and \code{sum = 22},
\begin{Code}
          5
         / \
        4   8
       /   / \
      11  13  4
     /  \      \
    7    2      1
\end{Code}
return true, as there exist a root-to-leaf path \code{5->4->11->2} which sum is 22.


\subsubsection{分析}
题目只要求返回\fn{true}或者\fn{false},因此不需要记录路径。

由于只需要求出一个结果,因此,当左、右任意一棵子树求到了满意结果,都可以及时return。

由于题目没有说节点的数据一定是正整数,必须要走到叶子节点才能判断,因此中途没法剪枝,只能进行朴素深搜。

\subsubsection{代码}

\begin{Code}
// LeetCode, Path Sum
class Solution {
public:
    bool hasPathSum(TreeNode *root, int sum) {
        if (root == nullptr) return false;

        if (root->left == nullptr && root->right == nullptr) { // leaf
            if (sum == root->val) return true;
            else return false;
        }
        if (hasPathSum(root->left, sum - root->val)) return true;
        if (hasPathSum(root->right, sum - root->val)) return true;
    }
};
\end{Code}


\subsubsection{相关题目}
\begindot
\item Path Sum II,见 \S \ref{sec:path-sum-ii}
\myenddot


\subsection{Path Sum II}
\label{sec:path-sum-ii}


\subsubsection{描述}
Given a binary tree and a sum, find all root-to-leaf paths where each path's sum equals the given sum.

For example:
Given the below binary tree and \code{sum = 22},
\begin{Code}
          5
         / \
        4   8
       /   / \
      11  13  4
     /  \    / \
    7    2  5   1
\end{Code}
return
\begin{Code}
[
   [5,4,11,2],
   [5,8,4,5]
]
\end{Code}


\subsubsection{分析}
跟上一题相比,本题是求路径本身。且要求出所有结果,左子树求到了满意结果,不能return,要接着求右子树。

\subsubsection{代码}

\begin{Code}
// LeetCode, Path Sum II
class Solution {
public:
    vector<vector<int> > pathSum(TreeNode *root, int sum) {
        vector<vector<int> > result;
        vector<int> cur; // 中间结果
        pathSum(root, sum, cur, result);
        return result;
    }
private:
    void pathSum(TreeNode *root, int gap, vector<int> &cur,
            vector<vector<int> > &result) {
        if (root == nullptr) return;

        cur.push_back(root->val);

        if (root->left == nullptr && root->right == nullptr) { // leaf
            if (gap == root->val) {
                result.push_back(cur);
            }
        }
        pathSum(root->left, gap - root->val, cur, result);
        pathSum(root->right, gap - root->val, cur, result);

        cur.pop_back();
    }
};
\end{Code}


\subsubsection{相关题目}
\begindot
\item Path Sum,见 \S \ref{sec:path-sum}
\myenddot


\subsection{Binary Tree Maximum Path Sum}
\label{sec:binary-tree-maximum-path-sum}


\subsubsection{描述}
Given a binary tree, find the maximum path sum.

The path may start and end at any node in the tree.
For example:
Given the below binary tree,
\begin{Code}
  1
 / \
2   3
\end{Code}
Return $6$.


\subsubsection{分析}
这题很难,路径可以从任意节点开始,到任意节点结束。

可以利用“最大连续子序列和”问题的思路,见第\S \ref{sec:maximum-subarray}节。如果说Array只有一个方向的话,那么Binary Tree其实只是左、右两个方向而已,我们需要比较两个方向上的值。

不过,Array可以从头到尾遍历,那么Binary Tree怎么办呢,我们可以采用Binary Tree最常用的dfs来进行遍历。先算出左右子树的结果L和R,如果L大于0,那么对后续结果是有利的,我们加上L,如果R大于0,对后续结果也是有利的,继续加上R。

\subsubsection{代码}

\begin{Code}
// LeetCode, Binary Tree Maximum Path Sum
class Solution {
public:
    int maxPathSum(TreeNode *root) {
        max = INT_MIN;
        dfs(root);
        return max;
    }
private:
    int max;
    int dfs(const TreeNode *root) {
        if (root == nullptr) return 0;
        int l = dfs(root->left);
        int r = dfs(root->right);
        int sum = root->val;
        if (l > 0) sum += l;
        if (r > 0) sum += r;
        max = max(max, sum);
        return max(r, l) > 0 ? max(r, l) + root->val : root->val;
    }
};
\end{Code}

注意,最后return的时候,只返回一个方向上的值,为什么?这是因为在递归中,只能向父节点返回,不可能存在L->root->R的路径,只可能是L->root或R->root。


\subsubsection{相关题目}
\begindot
\item Maximum Subarray,见 \S \ref{sec:maximum-subarray}
\myenddot


\section{Populating Next Right Pointers in Each Node} %%%%%%%%%%%%%%%%%%%%%%%%%%%%%%
\label{sec:populating-next-right-pointers-in-each-node}


\subsubsection{描述}
Given a binary tree
\begin{Code}
struct TreeLinkNode {
   int val;
   TreeLinkNode *left, *right, *next;
   TreeLinkNode(int x) : val(x), left(NULL), right(NULL), next(NULL) {}
};
\end{Code}

Populate each next pointer to point to its next right node. If there is no next right node, the next pointer should be set to \fn{NULL}.

Initially, all next pointers are set to \fn{NULL}.

Note:
\begindot
\item You may only use constant extra space.
\item You may assume that it is a perfect binary tree (ie, all leaves are at the same level, and every parent has two children).
\myenddot

For example,
Given the following perfect binary tree,
\begin{Code}
         1
       /  \
      2    3
     / \  / \
    4  5  6  7
\end{Code}

After calling your function, the tree should look like:
\begin{Code}
         1 -> NULL
       /  \
      2 -> 3 -> NULL
     / \  / \
    4->5->6->7 -> NULL
\end{Code}


\subsubsection{分析}


\subsubsection{代码}

\begin{Code}
// LeetCode, Populating Next Right Pointers in Each Node
class Solution {
public:
    void connect(TreeLinkNode *root) {
        connect(root, NULL);
    }
private:
    void connect(TreeLinkNode *root, TreeLinkNode *sibling) {
        if (root == nullptr)
            return;
        else
            root->next = sibling;

        connect(root->left, root->right);
        if (sibling)
            connect(root->right, sibling->left);
        else
            connect(root->right, nullptr);
    }
};
\end{Code}


\subsubsection{相关题目}
\begindot
\item Populating Next Right Pointers in Each Node II,见 \S \ref{sec:populating-next-right-pointers-in-each-node-ii}
\myenddot



\section{Populating Next Right Pointers in Each Node II} %%%%%%%%%%%%%%%%%%%%%%%%%%%%%%
\label{sec:populating-next-right-pointers-in-each-node-ii}


\subsubsection{描述}
Follow up for problem "Populating Next Right Pointers in Each Node".

What if the given tree could be any binary tree? Would your previous solution still work?

Note: You may only use constant extra space.

For example,
Given the following binary tree,
\begin{Code}
         1
       /  \
      2    3
     / \    \
    4   5    7
\end{Code}

After calling your function, the tree should look like:
\begin{Code}
         1 -> NULL
       /  \
      2 -> 3 -> NULL
     / \    \
    4-> 5 -> 7 -> NULL
\end{Code}


\subsubsection{分析}
要处理一个节点,可能需要最右边的兄弟节点,因此用广搜。

同时注意,这题的代码原封不动,可以解决上一题!

\subsubsection{代码}

\begin{Code}
// LeetCode, Populating Next Right Pointers in Each Node II
class Solution {
public:
    void connect(TreeLinkNode *root) {
        vector<vector<int> > result;
        if (root == nullptr)
            return;

        vector<TreeLinkNode*> nextLevel, currentLevel;
        currentLevel.push_back(root);
        while (!currentLevel.empty()) {
            while (!currentLevel.empty()) {
                TreeLinkNode* node = currentLevel.front();
                currentLevel.erase(currentLevel.begin());

                if (!currentLevel.empty())
                    node->next = currentLevel.front();
                else
                    node->next = nullptr;

                if (node->left != nullptr)
                    nextLevel.push_back(node->left);
                if (node->right != nullptr)
                    nextLevel.push_back(node->right);
            }
            swap(nextLevel, currentLevel); //!!! how to use swap
        }
    }
};
\end{Code}


\subsubsection{相关题目}
\begindot
\item Populating Next Right Pointers in Each Node,见 \S \ref{sec:populating-next-right-pointers-in-each-node}
\myenddot
