\subsubsection{内容简介}
本书的目标读者是准备去北美找工作的码农,也适用于在国内找工作的码农,以及刚接触ACM算法竞赛的新手。

本书包含了 LeetCode Online Judge(\myurl{http://leetcode.com/onlinejudge})所有题目的答案,
所有代码经过精心编写,编码规范良好,适合读者反复揣摩,模仿,甚至在纸上默写。

全书的代码,使用C++ 11的编写,并在 LeetCode Online Judge 上测试通过。本书中的代码规范,跟在公司中的工程规范略有不同,为了使代码短(方便迅速实现):

\begindot
\item 所有代码都是单一文件。这是因为一般OJ网站,提交代码的时候只有一个文本框,如果还是
按照标准做法,比如分为头文件.h和源代码.cpp,无法在网站上提交;

\item Shorter is better。能递归则一定不用栈;能用STL则一定不自己实现。

\item 不提倡防御式编程。不需要检查malloc()/new 返回的指针是否为NULL;不需要检查内部函数入口参数的有效性;使用纯C基于对象编程时,调用对象的成员方法,不需要检查对象自身是否为NULL。
\myenddot

本手册假定读者已经学过《数据结构》\footnote{《数据结构》,严蔚敏等著,清华大学出版社,
\myurl{http://book.douban.com/subject/2024655/}},
《算法》\footnote{《Algorithms》,Robert Sedgewick, Addison-Wesley Professional, \myurl{http://book.douban.com/subject/4854123/}}
这两门课,熟练掌握C++或Java。

\subsubsection{GitCafe地址}
本书是开源的,项目地址:\myurl{https://gitcafe.com/soulmachine/LeetCode}
